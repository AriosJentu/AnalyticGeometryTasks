\documentclass[fleqn, 14pt, a4paper, titlepage]{extarticle}
% \usepackage{styleclassic}
\usepackage{style}
\usepackage{titlepage}
\usepackage{symbols}

\rhead{\footnotesize Максимов П.А., Удалов А.А.}
\lhead{\footnotesize <<Аналитическая геометрия>>, задачи,  I семестр}

\renewcommand{\drawer}{}

\begin{document}

	\setcounter{page}{-1}
	\doctitlepage{
	\schoolfrom{ИНСТИТУТ МАТЕМАТИКИ И КОМПЬЮТЕРНЫХ ТЕХНОЛОГИЙ}{Департамент математического и компьютерного моделирования}
}{
	Максимов П.А.

	Удалов А.А.

	\vspace{110pt}
	
	\textbf{\large Сборник задач по дисциплине \\<<Аналитическая геометрия>>}

	\vspace{30pt}

	{\large I семестр}
}{2022}
	\thispagestyle{empty}

	\vspace*{-120pt}
	\tableofcontents

	%---------------------------------------------------------------------------------------------------------------
	%===============================================================================================================
	%---------------------------------------------------------------------------------------------------------------
	
	\renewcommand{\partname}{Раздел}

	%---------------------------------------------------------------------------------------------------------------
	%===============================================================================================================
	%---------------------------------------------------------------------------------------------------------------
	
	\newpart{Векторы, прямые и плоскости}

	%---------------------------------------------------------------------------------------------------------------
	%===============================================================================================================
	%---------------------------------------------------------------------------------------------------------------
	
	\input{parts/Vectors/vectors.tex}
	\subsection{Задачи}

	Для следующих пар двумерных векторов $\va, \vb = \bracs{x, y}$, найти их сумму и разность, и построить в системе координат $Oxy$:

	\begin{multicols}{2}
		\begin{enumerate}
			\setcounter{enumi}{\value{tasks}}

			\item \( \displaystyle \va = \bracs{7, -7}, ~ \vb = \bracs{-7, 5} \);
			\item \( \displaystyle \va = \bracs{-8, -3}, ~ \vb = \bracs{-2, 9} \);
			\item \( \displaystyle \va = \bracs{-1, -5}, ~ \vb = \bracs{-6, 0} \);
			\item \( \displaystyle \va = \bracs{0, 6}, ~ \vb = \bracs{2, -6} \);
			\item \( \displaystyle \va = \bracs{3, 9}, ~ \vb = \bracs{-9, 2} \);
			\item \( \displaystyle \va = \bracs{-6, -9}, ~ \vb = \bracs{2, 6} \);
			\item \( \displaystyle \va = \bracs{5, -1}, ~ \vb = \bracs{-1, 5} \);
			\item \( \displaystyle \va = \bracs{6, 0}, ~ \vb = \bracs{-6, -10} \);
			\item \( \displaystyle \va = \bracs{2, -8}, ~ \vb = \bracs{-5, 3} \);
			\item \( \displaystyle \va = \bracs{3, 1}, ~ \vb = \bracs{-5, 8} \);
			\item \( \displaystyle \va = \bracs{-5, 9}, ~ \vb = \bracs{5, 7} \);
			\item \( \displaystyle \va = \bracs{-3, -8}, ~ \vb = \bracs{-4, 3} \);
			\item \( \displaystyle \va = \bracs{-6, -8}, ~ \vb = \bracs{8, -4} \);
			\item \( \displaystyle \va = \bracs{-6, -7}, ~ \vb = \bracs{-3, 8} \);
			\item \( \displaystyle \va = \bracs{-5, -1}, ~ \vb = \bracs{-2, 7} \);
			\item \( \displaystyle \va = \bracs{9, -1}, ~ \vb = \bracs{-4, 9} \);
			\item \( \displaystyle \va = \bracs{-1, -1}, ~ \vb = \bracs{0, 9} \);
			\item \( \displaystyle \va = \bracs{2, 10}, ~ \vb = \bracs{-6, -5} \);
			\item \( \displaystyle \va = \bracs{10, 1}, ~ \vb = \bracs{-3, -2} \);
			\item \( \displaystyle \va = \bracs{-5, -4}, ~ \vb = \bracs{2, -4} \);
			\item \( \displaystyle \va = \bracs{4, 6}, ~ \vb = \bracs{10, 9} \);
			\item \( \displaystyle \va = \bracs{1, 2}, ~ \vb = \bracs{-8, 9} \);

			\setcounter{tasks}{\value{enumi}}
		\end{enumerate}
	\end{multicols}

	\vspace{15pt}
	Для следующих пар трехмерных векторов $\va, \vb = \bracs{x, y, z}$, найти их сумму и разность:

	\begin{enumerate}
		\setcounter{enumi}{\value{tasks}}

		\item \( \displaystyle \va = \bracs{-8, 3, 3}, ~ \vb = \bracs{10, 8, -5} \);
		\item \( \displaystyle \va = \bracs{5, -8, -4}, ~ \vb = \bracs{-5, -10, -6} \);
		\item \( \displaystyle \va = \bracs{-2, -5, 2}, ~ \vb = \bracs{6, -4, 6} \);
		\item \( \displaystyle \va = \bracs{10, 6, 10}, ~ \vb = \bracs{6, 0, 0} \);
		\item \( \displaystyle \va = \bracs{0, 1, 1}, ~ \vb = \bracs{4, -10, 9} \);
		\item \( \displaystyle \va = \bracs{-8, -7, -3}, ~ \vb = \bracs{-10, 8, -4} \);
		\item \( \displaystyle \va = \bracs{-1, 0, 2}, ~ \vb = \bracs{4, -5, -2} \);
		\item \( \displaystyle \va = \bracs{-1, 5, -9}, ~ \vb = \bracs{-4, -10, 3} \);
		\item \( \displaystyle \va = \bracs{6, -4, 2}, ~ \vb = \bracs{6, -2, -10} \);
		\item \( \displaystyle \va = \bracs{-6, -3, -2}, ~ \vb = \bracs{1, 1, 5} \);
		\item \( \displaystyle \va = \bracs{1, -9, 1}, ~ \vb = \bracs{5, 8, -2} \);
		\item \( \displaystyle \va = \bracs{-5, -1, 6}, ~ \vb = \bracs{3, 4, 3} \);
		\item \( \displaystyle \va = \bracs{-6, -7, 1}, ~ \vb = \bracs{2, 3, -3} \);
		\item \( \displaystyle \va = \bracs{10, 3, 0}, ~ \vb = \bracs{-4, -2, -6} \);
		\item \( \displaystyle \va = \bracs{-8, 2, 5}, ~ \vb = \bracs{-5, 7, 7} \);
		\item \( \displaystyle \va = \bracs{-5, 7, 10}, ~ \vb = \bracs{6, 6, -9} \);

		\setcounter{tasks}{\value{enumi}}
	\end{enumerate}

	\vspace{15pt}
	Найти значение вектора $\vc$, если известны следующие выражения:

	\begin{enumerate}
		\setcounter{enumi}{\value{tasks}}

			\item \( \displaystyle \vc = - 2 \cdot \va - 4 \cdot \vb, ~ \va = \bracs{2, 5, -4}, ~ \vb = \bracs{2, 5, 9} \);
			\item \( \displaystyle \vc = - 4 \cdot \va - 1 \cdot \vb, ~ \va = \bracs{10, -2, -3}, ~ \vb = \bracs{8, -6, -5} \);
			\item \( \displaystyle \vc = 7 \cdot \va + 3 \cdot \vb, ~ \va = \bracs{-5, -3, 4}, ~ \vb = \bracs{0, -7, 4} \);
			\item \( \displaystyle \vc = 7 \cdot \va - 3 \cdot \vb, ~ \va = \bracs{-3, -5, -5}, ~ \vb = \bracs{2, -3, -9} \);
			\item \( \displaystyle \vc = 8 \cdot \va - 4 \cdot \vb, ~ \va = \bracs{-7, 6, -9}, ~ \vb = \bracs{-1, 7, 2} \);
			\item \( \displaystyle \vc = 4 \cdot \va - 7 \cdot \vb, ~ \va = \bracs{-3, -7, 5}, ~ \vb = \bracs{-7, 9, 10} \);
			\item \( \displaystyle \vc = - 6 \cdot \va + 4 \cdot \vb, ~ \va = \bracs{-5, 1, -2}, ~ \vb = \bracs{0, -10, -7} \);
			\item \( \displaystyle \vc = 5 \cdot \va - 6 \cdot \vb, ~ \va = \bracs{10, -4, -3}, ~ \vb = \bracs{-1, 2, 5} \);
			\item \( \displaystyle \vc = - 4 \cdot \va - 2 \cdot \vb, ~ \va = \bracs{4, 4, -4}, ~ \vb = \bracs{1, 6, 7} \);
			\item \( \displaystyle \vc = 8 \cdot \va + \vb, ~ \va = \bracs{0, -10, -9}, ~ \vb = \bracs{7, -4, 1} \);
			\item \( \displaystyle \vc = - 5 \cdot \va + 6 \cdot \vb, ~ \va = \bracs{-9, -1, -2}, ~ \vb = \bracs{8, 10, -6} \);
			\item \( \displaystyle \vc = - \va - 7 \cdot \vb, ~ \va = \bracs{-2, -3, 1}, ~ \vb = \bracs{10, 2, 6} \);
			\item \( \displaystyle \vc = \va + 3 \cdot \vb, ~ \va = \bracs{7, 6, 8}, ~ \vb = \bracs{9, 9, 9} \);
			\item \( \displaystyle \vc = 7 \cdot \va - 3 \cdot \vb, ~ \va = \bracs{-5, 2, -8}, ~ \vb = \bracs{6, -1, 3} \);
			\item \( \displaystyle \vc = \va + 4 \cdot \vb, ~ \va = \bracs{-8, 5, 2}, ~ \vb = \bracs{5, 1, 1} \);
			\item \( \displaystyle \vc = 9 \cdot \va + 4 \cdot \vb, ~ \va = \bracs{6, -4, 10}, ~ \vb = \bracs{8, -9, -9} \);
			\item \( \displaystyle \vc = 9 \cdot \va + 7 \cdot \vb, ~ \va = \bracs{5, -10, -10}, ~ \vb = \bracs{-1, 5, 2} \);
			\item \( \displaystyle \vc = - 2 \cdot \va + 8 \cdot \vb, ~ \va = \bracs{2, 6, 10}, ~ \vb = \bracs{2, -10, -1} \);
			\item \( \displaystyle \vc = - \va + 8 \cdot \vb, ~ \va = \bracs{8, -3, -3}, ~ \vb = \bracs{8, -8, 5} \);
			\item \( \displaystyle \vc = - 4 \cdot \va - 3 \cdot \vb, ~ \va = \bracs{1, 2, 4}, ~ \vb = \bracs{0, -3, -2} \);

		\setcounter{tasks}{\value{enumi}}
	\end{enumerate}

	\pagebreak
	Построить нормированный базис $\ve_1, \ve_2$, в основании которого лежит следующая система векторов:

	\begin{multicols}{2}
		\begin{enumerate}
			\setcounter{enumi}{\value{tasks}}

			\item \( \displaystyle \vf_1 = \bracs{2, -9}, ~ \vf_2 = \bracs{-7, 7} \);
			\item \( \displaystyle \vf_1 = \bracs{8, 10}, ~ \vf_2 = \bracs{9, -4} \);
			\item \( \displaystyle \vf_1 = \bracs{5, 1}, ~ \vf_2 = \bracs{9, 4} \);
			\item \( \displaystyle \vf_1 = \bracs{8, -6}, ~ \vf_2 = \bracs{-6, 9} \);
			\item \( \displaystyle \vf_1 = \bracs{2, -8}, ~ \vf_2 = \bracs{-4, 4} \);
			\item \( \displaystyle \vf_1 = \bracs{-2, 5}, ~ \vf_2 = \bracs{7, -5} \);
			\item \( \displaystyle \vf_1 = \bracs{-3, 5}, ~ \vf_2 = \bracs{1, 3} \);
			\item \( \displaystyle \vf_1 = \bracs{-4, 3}, ~ \vf_2 = \bracs{-8, 8} \);
			\item \( \displaystyle \vf_1 = \bracs{-8, -6}, ~ \vf_2 = \bracs{1, 1} \);
			\item \( \displaystyle \vf_1 = \bracs{-4, 10}, ~ \vf_2 = \bracs{5, 3} \);
			\item \( \displaystyle \vf_1 = \bracs{9, -5}, ~ \vf_2 = \bracs{-2, 6} \);
			\item \( \displaystyle \vf_1 = \bracs{5, 9}, ~ \vf_2 = \bracs{-9, 10} \);
			\item \( \displaystyle \vf_1 = \bracs{-5, 4}, ~ \vf_2 = \bracs{6, -7} \);
			\item \( \displaystyle \vf_1 = \bracs{5, -2}, ~ \vf_2 = \bracs{-2, 4} \);
			\item \( \displaystyle \vf_1 = \bracs{-6, -5}, ~ \vf_2 = \bracs{3, 3} \);
			\item \( \displaystyle \vf_1 = \bracs{6, 1}, ~ \vf_2 = \bracs{9, 8} \);
			\item \( \displaystyle \vf_1 = \bracs{1, 4}, ~ \vf_2 = \bracs{7, 9} \);
			\item \( \displaystyle \vf_1 = \bracs{5, 8}, ~ \vf_2 = \bracs{8, -1} \);

			\setcounter{tasks}{\value{enumi}}
		\end{enumerate}
	\end{multicols}

	\vspace{15pt}
	Найти координаты вектора $\va$ из декартовой системы координат в линейном базисе $\ve_1, \ve_2$:

	\begin{enumerate}
		\setcounter{enumi}{\value{tasks}}

		\item \( \displaystyle \va = \bracs{-9, 9}, \quad \ve_1 = \bracs{-7, 7}, ~ \ve_2 = \bracs{4, 4} \);
		\item \( \displaystyle \va = \bracs{2, -7}, \quad \ve_1 = \bracs{5, 10}, ~ \ve_2 = \bracs{-5, 7} \);
		\item \( \displaystyle \va = \bracs{6, -4}, \quad \ve_1 = \bracs{-8, 2}, ~ \ve_2 = \bracs{-3, -4} \);
		\item \( \displaystyle \va = \bracs{8, -8}, \quad \ve_1 = \bracs{-9, 8}, ~ \ve_2 = \bracs{7, 10} \);
		\item \( \displaystyle \va = \bracs{-8, -3}, \quad \ve_1 = \bracs{8, 3}, ~ \ve_2 = \bracs{-9, -3} \);
		\item \( \displaystyle \va = \bracs{-6, 0}, \quad \ve_1 = \bracs{-1, -2}, ~ \ve_2 = \bracs{7, -9} \);
		\item \( \displaystyle \va = \bracs{-5, -1}, \quad \ve_1 = \bracs{-8, 2}, ~ \ve_2 = \bracs{1, -3} \);
		\item \( \displaystyle \va = \bracs{-6, 4}, \quad \ve_1 = \bracs{8, 9}, ~ \ve_2 = \bracs{1, -3} \);
		\item \( \displaystyle \va = \bracs{-3, 0}, \quad \ve_1 = \bracs{5, 8}, ~ \ve_2 = \bracs{-3, 1} \);
		\item \( \displaystyle \va = \bracs{-5, 3}, \quad \ve_1 = \bracs{-2, -5}, ~ \ve_2 = \bracs{2, -2} \);
		\item \( \displaystyle \va = \bracs{5, 8}, \quad \ve_1 = \bracs{1, 9}, ~ \ve_2 = \bracs{-5, -2} \);
		\item \( \displaystyle \va = \bracs{0, -3}, \quad \ve_1 = \bracs{5, 0}, ~ \ve_2 = \bracs{2, 1} \);
		\item \( \displaystyle \va = \bracs{2, 3}, \quad \ve_1 = \bracs{5, -6}, ~ \ve_2 = \bracs{-4, 1} \);
		\item \( \displaystyle \va = \bracs{9, 9}, \quad \ve_1 = \bracs{5, 5}, ~ \ve_2 = \bracs{2, 9} \);
		\item \( \displaystyle \va = \bracs{-10, 2}, \quad \ve_1 = \bracs{2, 10}, ~ \ve_2 = \bracs{-8, 5} \);
		\item \( \displaystyle \va = \bracs{5, -1}, \quad \ve_1 = \bracs{0, -9}, ~ \ve_2 = \bracs{-6, -5} \);
		\item \( \displaystyle \va = \bracs{4, 0}, \quad \ve_1 = \bracs{-6, 8}, ~ \ve_2 = \bracs{-4, 0} \);
		\item \( \displaystyle \va = \bracs{-6, -3}, \quad \ve_1 = \bracs{-6, -5}, ~ \ve_2 = \bracs{-10, 5} \);

		\setcounter{tasks}{\value{enumi}}
	\end{enumerate}

	\vspace{15pt}
	Найти длины двух векторов и угол между ними в двумерном пространстве:

	\begin{multicols}{2}
		\begin{enumerate}
			\setcounter{enumi}{\value{tasks}}

				\item \( \displaystyle \va = \bracs{1, -3}, ~ \vb = \bracs{3, -1} \);
				\item \( \displaystyle \va = \bracs{-1, -5}, ~ \vb = \bracs{5, 1} \);
				\item \( \displaystyle \va = \bracs{0, -2}, ~ \vb = \bracs{-1, 4} \);
				\item \( \displaystyle \va = \bracs{-3, 1}, ~ \vb = \bracs{-2, 4} \);
				\item \( \displaystyle \va = \bracs{-2, 5}, ~ \vb = \bracs{2, -4} \);
				\item \( \displaystyle \va = \bracs{3, 1}, ~ \vb = \bracs{0, 4} \);
				\item \( \displaystyle \va = \bracs{3, -2}, ~ \vb = \bracs{-4, -4} \);
				\item \( \displaystyle \va = \bracs{0, -2}, ~ \vb = \bracs{-5, 2} \);
				\item \( \displaystyle \va = \bracs{-1, 1}, ~ \vb = \bracs{3, -3} \);
				\item \( \displaystyle \va = \bracs{-5, 0}, ~ \vb = \bracs{1, -5} \);
				\item \( \displaystyle \va = \bracs{1, 4}, ~ \vb = \bracs{4, -2} \);
				\item \( \displaystyle \va = \bracs{-1, 2}, ~ \vb = \bracs{1, -1} \);
				\item \( \displaystyle \va = \bracs{2, -2}, ~ \vb = \bracs{4, 0} \);
				\item \( \displaystyle \va = \bracs{-2, -3}, ~ \vb = \bracs{-4, 5} \);
				\item \( \displaystyle \va = \bracs{3, 3}, ~ \vb = \bracs{-5, 2} \);
				\item \( \displaystyle \va = \bracs{-1, -5}, ~ \vb = \bracs{5, -4} \);

			\setcounter{tasks}{\value{enumi}}
		\end{enumerate}
	\end{multicols}

	\vspace{15pt}
	Найти скалярное произведение двух векторов в двумерном пространстве, если известны их длины и косинус угла между ними:
	
	\begin{enumerate}
		\setcounter{enumi}{\value{tasks}}
	
			\item \( \displaystyle \va \cdot \vb, ~ \babs{\va} = 3, ~ \babs{\vb} = 5, \quad \cos{\alpha} = -\frac{3}{5} \);
			\item \( \displaystyle \va \cdot \vb, ~ \babs{\va} = 3, ~ \babs{\vb} = 5, \quad \cos{\alpha} = \frac{4}{5} \);
			\item \( \displaystyle \va \cdot \vb, ~ \babs{\va} = 2, ~ \babs{\vb} = \sqrt{2}, \quad \cos{\alpha} = -\frac{2}{ \sqrt{8} } \);
			\item \( \displaystyle \va \cdot \vb, ~ \babs{\va} = \sqrt{8}, ~ \babs{\vb} = \sqrt{5}, \quad \cos{\alpha} = \frac{2}{ \sqrt{40} } \);
			\item \( \displaystyle \va \cdot \vb, ~ \babs{\va} = \sqrt{10}, ~ \babs{\vb} = \sqrt{5}, \quad \cos{\alpha} = \frac{1}{ \sqrt{50} } \);
			\item \( \displaystyle \va \cdot \vb, ~ \babs{\va} = \sqrt{5}, ~ \babs{\vb} = 5, \quad \cos{\alpha} = \frac{10}{ \sqrt{125} } \);
			\item \( \displaystyle \va \cdot \vb, ~ \babs{\va} = \sqrt{5}, ~ \babs{\vb} = \sqrt{34}, \quad \cos{\alpha} = -\frac{7}{ \sqrt{170} } \);
			\item \( \displaystyle \va \cdot \vb, ~ \babs{\va} = \sqrt{10}, ~ \babs{\vb} = 5, \quad \cos{\alpha} = -\frac{9}{ \sqrt{250} } \);
			\item \( \displaystyle \va \cdot \vb, ~ \babs{\va} = \sqrt{20}, ~ \babs{\vb} = \sqrt{17}, \quad \cos{\alpha} = \frac{12}{ \sqrt{340} } \);
			\item \( \displaystyle \va \cdot \vb, ~ \babs{\va} = \sqrt{20}, ~ \babs{\vb} = \sqrt{17}, \quad \cos{\alpha} = -\frac{14}{ \sqrt{340} } \);
			\item \( \displaystyle \va \cdot \vb, ~ \babs{\va} = 5, ~ \babs{\vb} = \sqrt{17}, \quad \cos{\alpha} = \frac{19}{ \sqrt{425} } \);
			\item \( \displaystyle \va \cdot \vb, ~ \babs{\va} = \sqrt{13}, ~ \babs{\vb} = \sqrt{34}, \quad \cos{\alpha} = \frac{21}{ \sqrt{442} } \);
			\item \( \displaystyle \va \cdot \vb, ~ \babs{\va} = \sqrt{32}, ~ \babs{\vb} = 4, \quad \cos{\alpha} = -\frac{16}{ \sqrt{512} } \);
			\item \( \displaystyle \va \cdot \vb, ~ \babs{\va} = 5, ~ \babs{\vb} = \sqrt{34}, \quad \cos{\alpha} = -\frac{29}{ \sqrt{850} } \);

		\setcounter{tasks}{\value{enumi}}
	\end{enumerate}

	\vspace{15pt}
	Найти скалярное произведение двух векторов:

	\begin{enumerate}
		\setcounter{enumi}{\value{tasks}}

		\item \( \displaystyle \va = \bracs{7, -5, -1}, ~ \vb = \bracs{-10, -3, 2} \);
		\item \( \displaystyle \va = \bracs{6, 3, -6}, ~ \vb = \bracs{6, -7, 7} \);
		\item \( \displaystyle \va = \bracs{10, 9, -4}, ~ \vb = \bracs{1, -7, -5} \);
		\item \( \displaystyle \va = \bracs{1, 3, -6}, ~ \vb = \bracs{-10, 10, 4} \);
		\item \( \displaystyle \va = \bracs{4, 7, -10}, ~ \vb = \bracs{10, -10, -5} \);
		\item \( \displaystyle \va = \bracs{4, -4, 3}, ~ \vb = \bracs{-7, 4, -5} \);
		\item \( \displaystyle \va = \bracs{5, 2, 2}, ~ \vb = \bracs{-4, 9, -4} \);
		\item \( \displaystyle \va = \bracs{-1, 6, -9}, ~ \vb = \bracs{-5, 5, -5} \);
		\item \( \displaystyle \va = \bracs{-4, 6, -10}, ~ \vb = \bracs{5, -9, -7} \);
		\item \( \displaystyle \va = \bracs{5, 8, 4}, ~ \vb = \bracs{-5, 4, -8} \);
		\item \( \displaystyle \va = \bracs{5, 3, 7}, ~ \vb = \bracs{-8, 8, 0} \);
		\item \( \displaystyle \va = \bracs{-10, -7, 9}, ~ \vb = \bracs{-4, -8, -2} \);
		\item \( \displaystyle \va = \bracs{9, 10, 6}, ~ \vb = \bracs{-3, 10, -7} \);
		\item \( \displaystyle \va = \bracs{10, -9, 6}, ~ \vb = \bracs{-8, 8, -7} \);
		\item \( \displaystyle \va = \bracs{-1, -3, 4}, ~ \vb = \bracs{-2, 4, 10} \);
		\item \( \displaystyle \va = \bracs{0, 3, 2}, ~ \vb = \bracs{10, -10, -4} \);

		\setcounter{tasks}{\value{enumi}}
	\end{enumerate}

	\vspace{15pt}
	Найти косинус угла между векторами, если известны их длины и скалярное произведение:

	\begin{enumerate}
		\setcounter{enumi}{\value{tasks}}

			\item \( \displaystyle \babs{\va} = 4, ~ \babs{\vb} = 2, ~ \va \cdot \vb = -8 \);
			\item \( \displaystyle \babs{\va} = 5, ~ \babs{\vb} = 3, ~ \va \cdot \vb = 0 \);
			\item \( \displaystyle \babs{\va} = 1, ~ \babs{\vb} = \sqrt{41}, ~ \va \cdot \vb = -4 \);
			\item \( \displaystyle \babs{\va} = 2, ~ \babs{\vb} = \sqrt{2}, ~ \va \cdot \vb = -2 \);
			\item \( \displaystyle \babs{\va} = 6, ~ \babs{\vb} = \sqrt{26}, ~ \va \cdot \vb = -14 \);
			\item \( \displaystyle \babs{\va} = \sqrt{45}, ~ \babs{\vb} = 1, ~ \va \cdot \vb = 5 \);
			\item \( \displaystyle \babs{\va} = \sqrt{10}, ~ \babs{\vb} = 5, ~ \va \cdot \vb = 13 \);
			\item \( \displaystyle \babs{\va} = \sqrt{20}, ~ \babs{\vb} = 5, ~ \va \cdot \vb = -20 \);
			\item \( \displaystyle \babs{\va} = \sqrt{29}, ~ \babs{\vb} = 5, ~ \va \cdot \vb = -26 \);
			\item \( \displaystyle \babs{\va} = \sqrt{35}, ~ \babs{\vb} = 6, ~ \va \cdot \vb = 6 \);
			\item \( \displaystyle \babs{\va} = \sqrt{5}, ~ \babs{\vb} = \sqrt{13}, ~ \va \cdot \vb = 3 \);
			\item \( \displaystyle \babs{\va} = \sqrt{13}, ~ \babs{\vb} = \sqrt{10}, ~ \va \cdot \vb = 9 \);
			\item \( \displaystyle \babs{\va} = \sqrt{14}, ~ \babs{\vb} = \sqrt{27}, ~ \va \cdot \vb = 12 \);
			\item \( \displaystyle \babs{\va} = \sqrt{26}, ~ \babs{\vb} = \sqrt{17}, ~ \va \cdot \vb = 19 \);
			\item \( \displaystyle \babs{\va} = \sqrt{34}, ~ \babs{\vb} = \sqrt{13}, ~ \va \cdot \vb = -1 \);
			\item \( \displaystyle \babs{\va} = \sqrt{35}, ~ \babs{\vb} = \sqrt{27}, ~ \va \cdot \vb = 21 \);
			\item \( \displaystyle \babs{\va} = \sqrt{33}, ~ \babs{\vb} = \sqrt{30}, ~ \va \cdot \vb = -26 \);
			\item \( \displaystyle \babs{\va} = \sqrt{27}, ~ \babs{\vb} = \sqrt{38}, ~ \va \cdot \vb = -24 \);
			\item \( \displaystyle \babs{\va} = \sqrt{38}, ~ \babs{\vb} = \sqrt{35}, ~ \va \cdot \vb = 2 \);
			\item \( \displaystyle \babs{\va} = \sqrt{41}, ~ \babs{\vb} = \sqrt{20}, ~ \va \cdot \vb = 26 \);
			
		\setcounter{tasks}{\value{enumi}}
	\end{enumerate}

	\pagebreak

	%---------------------------------------------------------------------------------------------------------------
	%===============================================================================================================
	%---------------------------------------------------------------------------------------------------------------
	
	\input{parts/Vectors/points.tex}
	\subsection{Задачи}

	Найти вектор перехода от точки $A$ к точке $B$:
	\begin{enumerate}
		\setcounter{enumi}{\value{tasks}}
			
			\item \( \displaystyle A = \pares{2, 10, 2}, ~ B = \pares{6, 3, 5} \);
			\item \( \displaystyle A = \pares{-6, 9, 1}, ~ B = \pares{1, -3, 2} \);
			\item \( \displaystyle A = \pares{6, 6, 5}, ~ B = \pares{-2, 8, -10} \);
			\item \( \displaystyle A = \pares{6, 1, -3}, ~ B = \pares{2, -4, 5} \);
			\item \( \displaystyle A = \pares{-3, 2, 5}, ~ B = \pares{10, 1, 0} \);
			\item \( \displaystyle A = \pares{-4, 8, -7}, ~ B = \pares{0, 6, -1} \);
			\item \( \displaystyle A = \pares{-2, -8, -1}, ~ B = \pares{-7, -6, -1} \);
			\item \( \displaystyle A = \pares{-4, -2, 7}, ~ B = \pares{-8, -2, 7} \);
			\item \( \displaystyle A = \pares{4, -3, -5}, ~ B = \pares{-7, -2, 2} \);
			\item \( \displaystyle A = \pares{9, -3, -3}, ~ B = \pares{-9, 9, 4} \);
			\item \( \displaystyle A = \pares{-10, 10, 2}, ~ B = \pares{-7, -1, -8} \);
			\item \( \displaystyle A = \pares{-2, 10, -4}, ~ B = \pares{-5, 6, 0} \);
			\item \( \displaystyle A = \pares{3, 0, -8}, ~ B = \pares{6, 0, 8} \);
			\item \( \displaystyle A = \pares{6, -7, -8}, ~ B = \pares{-9, 7, 4} \);
			\item \( \displaystyle A = \pares{-8, 0, -10}, ~ B = \pares{4, 7, 0} \);
			\item \( \displaystyle A = \pares{-9, 7, 4}, ~ B = \pares{4, -9, -1} \);
			\item \( \displaystyle A = \pares{-7, -10, -9}, ~ B = \pares{10, 3, 10} \);
			\item \( \displaystyle A = \pares{-4, 1, -8}, ~ B = \pares{-1, 2, -5} \);
			\item \( \displaystyle A = \pares{-5, -6, -8}, ~ B = \pares{8, 5, 6} \);
			\item \( \displaystyle A = \pares{1, 5, 8}, ~ B = \pares{4, -3, 3} \);

		\setcounter{tasks}{\value{enumi}}
	\end{enumerate}

	\vspace{15pt}
	Найти в двумерном пространстве положение точки $C$, полученной делением отрезка $AB$ в соотношении $\lambda = \dfrac{AC}{CB}$:

	\begin{enumerate}
		\setcounter{enumi}{\value{tasks}}

		\item \( \displaystyle A = \pares{2, 7}, ~ B = \pares{1, -10}, ~ \lambda = \frac{1}{2} \);
		\item \( \displaystyle A = \pares{-1, 1}, ~ B = \pares{6, -9}, ~ \lambda = \frac{1}{3} \);
		\item \( \displaystyle A = \pares{9, 2}, ~ B = \pares{-10, -9}, ~ \lambda = \frac{1}{3} \);
		\item \( \displaystyle A = \pares{0, 3}, ~ B = \pares{4, -9}, ~ \lambda = \frac{1}{3} \);
		\item \( \displaystyle A = \pares{-8, -6}, ~ B = \pares{2, -1}, ~ \lambda = \frac{2}{3} \);
		\item \( \displaystyle A = \pares{7, -3}, ~ B = \pares{-1, -4}, ~ \lambda = \frac{2}{3} \);
		\item \( \displaystyle A = \pares{-3, -1}, ~ B = \pares{7, 6}, ~ \lambda = \frac{1}{4} \);
		\item \( \displaystyle A = \pares{-9, 9}, ~ B = \pares{0, -1}, ~ \lambda = \frac{3}{4} \);
		\item \( \displaystyle A = \pares{-7, -10}, ~ B = \pares{-7, 0}, ~ \lambda = \frac{3}{4} \);
		\item \( \displaystyle A = \pares{8, 0}, ~ B = \pares{0, -5}, ~ \lambda = \frac{1}{5} \);
		\item \( \displaystyle A = \pares{-7, 0}, ~ B = \pares{9, 8}, ~ \lambda = \frac{2}{5} \);
		\item \( \displaystyle A = \pares{1, 5}, ~ B = \pares{2, 2}, ~ \lambda = \frac{1}{6} \);
		\item \( \displaystyle A = \pares{0, 3}, ~ B = \pares{4, 6}, ~ \lambda = \frac{5}{6} \);
		\item \( \displaystyle A = \pares{-6, 8}, ~ B = \pares{1, 3}, ~ \lambda = \frac{3}{7} \);
		\item \( \displaystyle A = \pares{6, -7}, ~ B = \pares{-1, -2}, ~ \lambda = \frac{4}{7} \);
		\item \( \displaystyle A = \pares{7, 7}, ~ B = \pares{-3, 6}, ~ \lambda = \frac{5}{7} \);
		\item \( \displaystyle A = \pares{7, -2}, ~ B = \pares{4, -3}, ~ \lambda = \frac{5}{8} \);
		\item \( \displaystyle A = \pares{-4, -5}, ~ B = \pares{-3, 8}, ~ \lambda = \frac{4}{9} \);

		\setcounter{tasks}{\value{enumi}}
	\end{enumerate}

	\vspace{15pt}
	Найти в трехмерном пространстве положение точки $C$, полученной делением отрезка $AB$ в соотношении $\lambda = \dfrac{AC}{CB}$:
	\begin{enumerate}
		\setcounter{enumi}{\value{tasks}}

		\item \( \displaystyle A = \pares{-6, -1, 7}, ~ B = \pares{-7, -8, 9}, ~ \lambda = \frac{1}{2} \);
		\item \( \displaystyle A = \pares{-1, -7, 0}, ~ B = \pares{-2, -5, 5}, ~ \lambda = \frac{1}{2} \);
		\item \( \displaystyle A = \pares{2, 0, -7}, ~ B = \pares{6, -10, 10}, ~ \lambda = \frac{1}{3} \);
		\item \( \displaystyle A = \pares{-6, 10, -5}, ~ B = \pares{10, -5, -3}, ~ \lambda = \frac{2}{3} \);
		\item \( \displaystyle A = \pares{-10, -5, 0}, ~ B = \pares{-2, -1, 4}, ~ \lambda = \frac{1}{4} \);
		\item \( \displaystyle A = \pares{-1, -3, -3}, ~ B = \pares{6, 4, -9}, ~ \lambda = \frac{1}{5} \);
		\item \( \displaystyle A = \pares{3, -1, 0}, ~ B = \pares{8, 3, -3}, ~ \lambda = \frac{4}{5} \);
		\item \( \displaystyle A = \pares{6, 9, -6}, ~ B = \pares{6, 0, 10}, ~ \lambda = \frac{1}{6} \);
		\item \( \displaystyle A = \pares{10, 8, 10}, ~ B = \pares{-10, -4, 5}, ~ \lambda = \frac{1}{7} \);
		\item \( \displaystyle A = \pares{-6, 7, 3}, ~ B = \pares{6, -9, -7}, ~ \lambda = \frac{4}{7} \);
		\item \( \displaystyle A = \pares{-4, 3, -5}, ~ B = \pares{0, -9, -9}, ~ \lambda = \frac{5}{7} \);
		\item \( \displaystyle A = \pares{2, 5, -3}, ~ B = \pares{5, 7, 8}, ~ \lambda = \frac{5}{8} \);

		\setcounter{tasks}{\value{enumi}}
	\end{enumerate}	

	\pagebreak

	%---------------------------------------------------------------------------------------------------------------
	%===============================================================================================================
	%---------------------------------------------------------------------------------------------------------------
	
	\input{parts/Vectors/lines.tex}
	\subsection{Задачи}

	Построить каноническое и параметрическое уравнения прямой, проходящей через точки в двумерном пространстве:
	\begin{multicols}{2}
		\begin{enumerate}
			\setcounter{enumi}{\value{tasks}}
				
				\item \( \displaystyle B = \pares{1, -1}, ~ N = \pares{-4, 2} \);
				\item \( \displaystyle R = \pares{-8, -2}, ~ I = \pares{8, -7} \);
				\item \( \displaystyle P = \pares{6, -7}, ~ C = \pares{10, -6} \);
				\item \( \displaystyle S = \pares{0, 10}, ~ M = \pares{4, 10} \);
				\item \( \displaystyle F = \pares{8, 9}, ~ L = \pares{-10, -7} \);
				\item \( \displaystyle V = \pares{-5, 8}, ~ E = \pares{-8, -4} \);
				\item \( \displaystyle L = \pares{-9, 1}, ~ F = \pares{-5, -6} \);
				\item \( \displaystyle H = \pares{3, 9}, ~ G = \pares{-1, -1} \);
				\item \( \displaystyle A = \pares{7, 5}, ~ O = \pares{-4, -2} \);
				\item \( \displaystyle H = \pares{-4, 7}, ~ R = \pares{-9, -1} \);
				\item \( \displaystyle Q = \pares{6, 0}, ~ K = \pares{7, -8} \);
				\item \( \displaystyle D = \pares{-4, 9}, ~ Y = \pares{-5, -2} \);
				\item \( \displaystyle J = \pares{-1, 0}, ~ S = \pares{-8, -5} \);
				\item \( \displaystyle Y = \pares{5, 5}, ~ Y = \pares{-7, -4} \);
				\item \( \displaystyle A = \pares{-2, 0}, ~ X = \pares{6, -9} \);
				\item \( \displaystyle J = \pares{5, 4}, ~ K = \pares{1, -5} \);
				\item \( \displaystyle G = \pares{-6, 4}, ~ H = \pares{5, -5} \);
				\item \( \displaystyle S = \pares{6, 9}, ~ N = \pares{4, 8} \);
				\item \( \displaystyle G = \pares{-6, -8}, ~ F = \pares{-8, 7} \);
				\item \( \displaystyle Q = \pares{7, 6}, ~ W = \pares{8, -3} \);

			\setcounter{tasks}{\value{enumi}}
		\end{enumerate}
	\end{multicols}

	\vspace{15pt}
	Построить параметрическое уравнение для следующих прямых:

	\begin{multicols}{2}
		\begin{enumerate}
			\setcounter{enumi}{\value{tasks}}
				
				\item \( \displaystyle y = \frac{6+x}{-4} \);
				\item \( \displaystyle x + 3 = \frac{-5+y}{-4} \);
				\item \( \displaystyle \frac{7+x}{6} = \frac{y+9}{16} \);
				\item \( \displaystyle \frac{6+x}{-1} = \frac{y-6}{-5} \);
				\item \( \displaystyle \frac{x-3}{6} = \frac{y-1}{-6} \);
				\item \( \displaystyle \frac{x+10}{9} = \frac{y+9}{18} \);
				\item \( \displaystyle \frac{-5+x}{-2} = \frac{y+2}{2} \);
				\item \( \displaystyle \frac{x+3}{4} = \frac{y-4}{-8} \);
				\item \( \displaystyle \frac{-4+x}{1} = \frac{-10+y}{-10} \);
				\item \( \displaystyle \frac{10+x}{8} = \frac{y+5}{12} \);
				\item \( \displaystyle \frac{x-2}{-2} = \frac{y-2}{8} \);
				\item \( \displaystyle \frac{x+5}{11} = \frac{6+y}{3} \);

			\setcounter{tasks}{\value{enumi}}
		\end{enumerate}
	\end{multicols}

	\vspace{10pt}
	\begin{multicols}{2}
		\begin{enumerate}
			\setcounter{enumi}{\value{tasks}}

				\item \( \displaystyle x-8 = 9-y = \frac{z-5}{3} \);
				\item \( \displaystyle \frac{x-2}{-10} = \frac{-8+y}{-8} = \frac{z+5}{-3} \);
				\item \( \displaystyle \frac{-1+x}{3} = \frac{9+y}{3} = \frac{z-5}{-1} \);
				\item \( \displaystyle \frac{-3+x}{2} = \frac{y-7}{-17} = \frac{-2+z}{-8} \);
				\item \( \displaystyle \frac{-9+x}{-15} = \frac{3+y}{11} = \frac{z-5}{-10} \);
				\item \( \displaystyle \frac{3+x}{0} = \frac{y+4}{8} = \frac{4+z}{8} \);
				\item \( \displaystyle \frac{x-1}{0} = \frac{-3+y}{4} = \frac{z-10}{-14} \);
				\item \( \displaystyle \frac{x-6}{-4} = \frac{y-4}{-12} = \frac{z+1}{0} \);

			\setcounter{tasks}{\value{enumi}}
		\end{enumerate}
	\end{multicols}

	\vspace{15pt}
	Построить вектор нормали для следующих прямых на плоскости:

	\begin{multicols}{2}
		\begin{enumerate}
			\setcounter{enumi}{\value{tasks}}

				\item \( \displaystyle y = 4 - x \)
				\item \( \displaystyle y = 5 - 8x \)
				\item \( \displaystyle y = 7x + 5 \)
				\item \( \displaystyle y = 4x - 3 \)

			\setcounter{tasks}{\value{enumi}}
		\end{enumerate}
	\end{multicols}

	\vspace{10pt}
	\begin{multicols}{2}
		\begin{enumerate}
			\setcounter{enumi}{\value{tasks}}

				\item \( \displaystyle y = \frac{3}{5} x + \frac{41}{5} \)
				\item \( \displaystyle y = \frac{3}{4} x - \frac{5}{4} \)
				\item \( \displaystyle y = \frac{3}{4} x + \frac{1}{4} \)
				\item \( \displaystyle y = \frac{3}{5} x + \frac{3}{5} \)

			\setcounter{tasks}{\value{enumi}}
		\end{enumerate}
	\end{multicols}

	\vspace{10pt}
	\begin{multicols}{2}
		\begin{enumerate}
			\setcounter{enumi}{\value{tasks}}

				\item \( \displaystyle \frac{9+x}{17} = \frac{y+8}{4} \);
				\item \( \displaystyle \frac{4+x}{13} = \frac{y+7}{14} \);
				\item \( \displaystyle \frac{5+x}{0} = \frac{-4+y}{5} \);
				\item \( \displaystyle \frac{-3+x}{-7} = \frac{-1+y}{0} \);
				\item \( \displaystyle \frac{x-7}{-13} = \frac{-1+y}{2} \);
				\item \( \displaystyle \frac{x-10}{-16} = \frac{y}{2} \);
				\item \( \displaystyle \frac{-8+x}{-12} = \frac{y-7}{1} \);
				\item \( \displaystyle \frac{9+x}{19} = \frac{6+y}{7} \);

			\setcounter{tasks}{\value{enumi}}
		\end{enumerate}
	\end{multicols}

	\vspace{10pt}
	\begin{multicols}{2}
		\begin{enumerate}
			\setcounter{enumi}{\value{tasks}}

				\item \( \displaystyle \left\lbrace \begin{aligned}
							x &= t - 3 \\
							y &= 1
						\end{aligned} \right. \);
				\item \( \displaystyle \left\lbrace \begin{aligned}
							x &= 1 \\
							y &= 4 + t
						\end{aligned} \right. \);
				\item \( \displaystyle \left\lbrace \begin{aligned}
							x &= 9 - 9t \\
							y &= -3 - t
						\end{aligned} \right. \);
				\item \( \displaystyle \left\lbrace \begin{aligned}
							x &= t - 3 \\
							y &= -7 + 5t
						\end{aligned} \right. \);

			\setcounter{tasks}{\value{enumi}}
		\end{enumerate}
	\end{multicols}

	\vspace{15pt}
	Построить каноническое и параметрическое уравнения прямой, проходящей через точку $M$, перпендикулярной вектору $\vn$:

	\begin{multicols}{2}
		\begin{enumerate}
			\setcounter{enumi}{\value{tasks}}
					
				\item \( \displaystyle M = \pares{7, -4}, ~ \vn = \bracs{1, -6} \);
				\item \( \displaystyle M = \pares{-2, 8}, ~ \vn = \bracs{-6, 6} \);
				\item \( \displaystyle M = \pares{4, 10}, ~ \vn = \bracs{-1, -2} \);
				\item \( \displaystyle M = \pares{4, -6}, ~ \vn = \bracs{2, 5} \);
				\item \( \displaystyle M = \pares{-9, 7}, ~ \vn = \bracs{-1, 4} \);
				\item \( \displaystyle M = \pares{-4, 1}, ~ \vn = \bracs{0, -5} \);
				\item \( \displaystyle M = \pares{5, -1}, ~ \vn = \bracs{8, -8} \);
				\item \( \displaystyle M = \pares{2, -2}, ~ \vn = \bracs{-3, 3} \);
				\item \( \displaystyle M = \pares{-9, 6}, ~ \vn = \bracs{-2, 4} \);
				\item \( \displaystyle M = \pares{10, 1}, ~ \vn = \bracs{-1, 5} \);
				\item \( \displaystyle M = \pares{1, 2}, ~ \vn = \bracs{-2, 10} \);
				\item \( \displaystyle M = \pares{-2, 2}, ~ \vn = \bracs{4, 2} \);
				\item \( \displaystyle M = \pares{-9, 7}, ~ \vn = \bracs{8, -4} \);
				\item \( \displaystyle M = \pares{-10, -8}, ~ \vn = \bracs{5, 8} \);
				\item \( \displaystyle M = \pares{-8, 7}, ~ \vn = \bracs{-9, 5} \);
				\item \( \displaystyle M = \pares{-9, 9}, ~ \vn = \bracs{6, 8} \);
				\item \( \displaystyle M = \pares{-7, 6}, ~ \vn = \bracs{3, -6} \);
				\item \( \displaystyle M = \pares{10, -3}, ~ \vn = \bracs{-4, 6} \);
				\item \( \displaystyle M = \pares{9, -5}, ~ \vn = \bracs{-8, 1} \);
				\item \( \displaystyle M = \pares{7, -10}, ~ \vn = \bracs{-2, 2} \);

			\setcounter{tasks}{\value{enumi}}
		\end{enumerate}
	\end{multicols}

	\pagebreak
	Построить каноническое и параметрическое уравнения прямой, проходящей через точку $O$, построенной по направлению вектора $\vv$ в трехмерном пространстве:

	\begin{enumerate}
		\setcounter{enumi}{\value{tasks}}

			\item \( \displaystyle O = \pares{-5, -6, -9}, ~ \vv = \bracs{-1, -3, -10} \);
			\item \( \displaystyle O = \pares{-2, 6, -1}, ~ \vv = \bracs{0, -5, 4} \);
			\item \( \displaystyle O = \pares{9, -2, 6}, ~ \vv = \bracs{-4, 9, 7} \);
			\item \( \displaystyle O = \pares{10, 3, 7}, ~ \vv = \bracs{7, -3, -9} \);
			\item \( \displaystyle O = \pares{4, 4, -10}, ~ \vv = \bracs{4, 4, -9} \);
			\item \( \displaystyle O = \pares{0, 0, 4}, ~ \vv = \bracs{-5, -1, 2} \);
			\item \( \displaystyle O = \pares{1, 10, -9}, ~ \vv = \bracs{4, -10, 8} \);
			\item \( \displaystyle O = \pares{-3, 9, -10}, ~ \vv = \bracs{7, -8, 10} \);
			\item \( \displaystyle O = \pares{-10, -1, 0}, ~ \vv = \bracs{6, 3, 8} \);
			\item \( \displaystyle O = \pares{-4, 0, 3}, ~ \vv = \bracs{4, 10, 5} \);

		\setcounter{tasks}{\value{enumi}}
	\end{enumerate}

	\vspace{15pt}
	Найти направляющие вектора и минимальный угол между прямыми, или показать, что они параллельны:

	\begin{multicols}{2}
		\begin{enumerate}
			\setcounter{enumi}{\value{tasks}}

				\item \( \displaystyle y = 6 - x, ~ y = 2 + 2x \);
				\item \( \displaystyle y = x + 2, ~ y = 2x - 5 \);
				\item \( \displaystyle y = 5x + 4, ~ y = 3x - 4  \);
				\item \( \displaystyle y = 4 - 7x, ~ y = 4x - 7 \);

			\setcounter{tasks}{\value{enumi}}
		\end{enumerate}
	\end{multicols}

	\vspace{10pt}
	\begin{enumerate}
		\setcounter{enumi}{\value{tasks}}

			\item \( \displaystyle \frac{-8+x}{-2} = \frac{y-9}{-11}, \quad \frac{-2+x}{5} = \frac{y}{7} \);
			\item \( \displaystyle \frac{x-4}{-9} = \frac{5+y}{7}, \quad \frac{5+x}{11} = \frac{4+y}{5} \);
			\item \( \displaystyle \frac{1+x}{-5} = \frac{2+y}{-6}, \quad \frac{-6+x}{4} = \frac{1+y}{-6} \);
			\item \( \displaystyle \frac{x-7}{3} = \frac{y-8}{-8}, \quad \frac{x}{-5} = \frac{y+9}{4} \);
			\item \( \displaystyle \frac{x-9}{1} = \frac{-8+y}{2}, \quad \frac{x+6}{5} = \frac{9+y}{2} \);
			\item \( \displaystyle \frac{x-7}{5} = \frac{y+4}{2}, \quad \frac{3-x}{-5} = \frac{-7+y}{-4} \);

		\setcounter{tasks}{\value{enumi}}
	\end{enumerate}

	\vspace{10pt}
	\begin{enumerate}
		\setcounter{enumi}{\value{tasks}}

			\item \( \displaystyle \left\lbrace \begin{aligned}
						x &= 6-8t \\
						y &= -11t+3
					\end{aligned} \right., \quad \left\lbrace \begin{aligned}
						x &= -4-3t \\
						y &= 7
					\end{aligned} \right. \);
			\item \( \displaystyle \left\lbrace \begin{aligned}
						x &= -6-4t \\
						y &= 1-6t
					\end{aligned} \right., \quad \left\lbrace \begin{aligned}
						x &= 7-9t \\
						y &= -4-3t
					\end{aligned} \right. \);
			\item \( \displaystyle \left\lbrace \begin{aligned}
						x &= 6t-7 \\
						y &= 4-8t
					\end{aligned} \right., \quad \left\lbrace \begin{aligned}
						x &= -10+12t \\
						y &= -8+5t
					\end{aligned} \right. \);
			\item \( \displaystyle \left\lbrace \begin{aligned}
						x &= -7+11t \\
						y &= -5t-1
					\end{aligned} \right., \quad \left\lbrace \begin{aligned}
						x &= 4 \\
						y &= 4t-7
					\end{aligned} \right. \);

		\setcounter{tasks}{\value{enumi}}
	\end{enumerate}

	\vspace{10pt}
	\begin{enumerate}
		\setcounter{enumi}{\value{tasks}}

			\item \( \displaystyle \frac{-4+x}{1} = \frac{-5+y}{-8} = \frac{8+z}{7}, \quad \frac{x-7}{-6} = \frac{-6+y}{1} = \frac{z-1}{0} \);
			\item \( \displaystyle \frac{-1+x}{7} = \frac{6+y}{-1} = \frac{1+z}{-3}, \quad \frac{x-2}{-3} = \frac{y+9}{6} = \frac{2+z}{2} \);
			\item \( \displaystyle \frac{x}{-3} = \frac{5+y}{-4} = \frac{z}{8}, \quad \frac{10+x}{7} = \frac{y-2}{8} = \frac{-10+z}{-4} \);
			\item \( \displaystyle \frac{x+5}{4} = \frac{-4+y}{-5} = \frac{z-9}{-8}, \quad \frac{-5+x}{2} = \frac{-9+y}{1} = \frac{-3+z}{7} \);
			\item \( \displaystyle \frac{x-6}{-1} = \frac{y-10}{-9} = \frac{z+6}{7}, \quad \frac{-5+x}{1} = \frac{y-6}{-2} = \frac{z-7}{-8} \);
			\item \( \displaystyle \frac{-8+x}{-4} = \frac{1+y}{5} = \frac{z+9}{3}, \quad \frac{2+x}{4} = \frac{y+3}{-6} = \frac{z-1}{1} \);
			\item \( \displaystyle \frac{x+6}{4} = \frac{y-1}{3} = \frac{-10+z}{-6}, \quad \frac{x-6}{-6} = \frac{1+y}{-4} = \frac{8+z}{5} \);
			\item \( \displaystyle \frac{x+4}{7} = \frac{-1+y}{8} = \frac{z-2}{7}, \quad \frac{-8+x}{-8} = \frac{10+y}{0} = \frac{3+z}{2} \);
			\item \( \displaystyle \frac{x+2}{8} = \frac{3+y}{5} = \frac{z+5}{-5}, \quad \frac{x+7}{6} = \frac{-1+y}{-1} = \frac{z-4}{0} \);
			\item \( \displaystyle \frac{x}{-3} = \frac{7+y}{4} = \frac{z+9}{1}, \quad \frac{-3+x}{-7} = \frac{-7+y}{-4} = \frac{-9+z}{-9} \);
			\item \( \displaystyle \frac{x+3}{-6} = \frac{-4+y}{-2} = \frac{z}{3}, \quad \frac{-5+x}{-7} = \frac{1+y}{7} = \frac{4+z}{-4} \);
			\item \( \displaystyle \frac{-5+x}{0} = \frac{y+3}{3} = \frac{-3+z}{2}, \quad \frac{-6+x}{-6} = \frac{4+y}{8} = \frac{z+10}{1} \);
			\item \( \displaystyle \frac{10+x}{-5} = \frac{y-3}{-2} = \frac{z-5}{-3}, \quad \frac{-4+x}{0} = \frac{y}{3} = \frac{z}{5} \);
			\item \( \displaystyle \frac{10+x}{6} = \frac{y-5}{-3} = \frac{z}{-5}, \quad \frac{-7+x}{2} = \frac{-3+y}{1} = \frac{z+9}{8} \);
			\item \( \displaystyle \frac{x-10}{-4} = \frac{y+5}{-1} = \frac{5+z}{-4}, \quad \frac{-7+x}{-6} = \frac{4+y}{-2} = \frac{z+6}{5} \);
			\item \( \displaystyle \frac{-1+x}{-2} = \frac{y-3}{0} = \frac{10+z}{7}, \quad \frac{5+x}{1} = \frac{y-1}{5} = \frac{-2+z}{6} \);

		\setcounter{tasks}{\value{enumi}}
	\end{enumerate}

	\vspace{15pt}
	Построить прямую, проходящую через точку $M$, поверную на угол $\alpha$ относительно прямой $l$:

	\begin{enumerate}
		\setcounter{enumi}{\value{tasks}}

			\item \( \displaystyle M = \left( 1, -3 \right), ~ \alpha = -\dfrac{3\pi}{4}, \quad l: \frac{-3+x}{-13} = \frac{y+10}{10} \);
			\item \( \displaystyle M = \left( -1, 1 \right), ~ \alpha = -\dfrac{5\pi}{3}, \quad l: \frac{9+x}{9} = \frac{y+4}{7} \);
			\item \( \displaystyle M = \left( -3, -1 \right), ~ \alpha = \dfrac{7\pi}{4}, \quad l: \frac{-2+x}{6} = \frac{5+y}{15} \);
			\item \( \displaystyle M = \left( 9, -4 \right), ~ \alpha = -\pi, \quad l: \frac{x-5}{-4} = \frac{y+10}{19} \);
			\item \( \displaystyle M = \left( -7, 9 \right), ~ \alpha = \dfrac{7\pi}{4}, \quad l: \frac{x-7}{3} = \frac{y+6}{0} \);
			\item \( \displaystyle M = \left( 3, 9 \right), ~ \alpha = -\dfrac{2\pi}{3}, \quad l: \frac{x+8}{7} = \frac{y+7}{13} \);
			\item \( \displaystyle M = \left( -2, 3 \right), ~ \alpha = \dfrac{7\pi}{6}, \quad l: \frac{x+5}{10} = \frac{y+6}{15} \);
			\item \( \displaystyle M = \left( -5, 6 \right), ~ \alpha = -\dfrac{3\pi}{2}, \quad l: \frac{6+x}{13} = \frac{10+y}{2} \);
			\item \( \displaystyle M = \left( -6, 4 \right), ~ \alpha = \dfrac{7\pi}{6}, \quad l: \frac{x-1}{-10} = \frac{-8+y}{-6} \);
			\item \( \displaystyle M = \left( 5, 3 \right), ~ \alpha = \dfrac{4\pi}{3}, \quad l: \frac{x-5}{-2} = \frac{y+5}{9} \);
			\item \( \displaystyle M = \left( 8, -6 \right), ~ \alpha = \pi, \quad l: \frac{-7+x}{-5} = \frac{y-9}{1} \);
			\item \( \displaystyle M = \left( 0, 1 \right), ~ \alpha = -\dfrac{3\pi}{2}, \quad l: \frac{-8+x}{-14} = \frac{5+y}{10} \);

		\setcounter{tasks}{\value{enumi}}
	\end{enumerate}

	\vspace{15pt}
	Найти расстояние от точки до прямой:

	\begin{enumerate}
		\setcounter{enumi}{\value{tasks}}
			
			\item \( \displaystyle B = \left( 6, 3 \right), \quad \frac{x+8}{18} = \frac{-10+y}{-17} \);
			\item \( \displaystyle J = \left( -8, -8 \right), \quad \frac{-4+x}{6} = \frac{7+y}{8} \);
			\item \( \displaystyle W = \left( -4, 3 \right), \quad \frac{-6+x}{-10} = \frac{y+9}{12} \);
			\item \( \displaystyle O = \left( -2, 10 \right), \quad \frac{-2+x}{-9} = \frac{y+5}{13} \);
			\item \( \displaystyle A = \left( -9, -10 \right), \quad \frac{x-8}{-14} = \frac{10+y}{10} \);
			\item \( \displaystyle B = \left( 4, 1 \right), \quad \frac{x+1}{-9} = \frac{8+y}{14} \);
			\item \( \displaystyle V = \left( 9, 0 \right), \quad \frac{-7+x}{2} = \frac{y+4}{-3} \);
			\item \( \displaystyle Z = \left( 4, -7 \right), \quad \frac{x}{1} = \frac{5+y}{13} \);
			\item \( \displaystyle C = \left( -9, -10 \right), \quad \frac{-5+x}{4} = \frac{y-5}{-13} \);
			\item \( \displaystyle M = \left( -3, 6 \right), \quad \frac{4+x}{12} = \frac{y-5}{2} \);
			\item \( \displaystyle P = \left( -1, 3 \right), \quad \frac{-2+x}{-10} = \frac{1+y}{3} \);
			\item \( \displaystyle V = \left( -3, 1 \right), \quad \frac{7+x}{12} = \frac{-6+y}{-5} \);
			\item \( \displaystyle X = \left( -6, -2 \right), \quad \frac{-10+x}{-16} = \frac{y-4}{-9} \);
			\item \( \displaystyle W = \left( 10, 6 \right), \quad \frac{x-4}{2} = \frac{4+y}{-1} \);
			\item \( \displaystyle G = \left( 8, -5 \right), \quad \frac{x-7}{-16} = \frac{y+9}{4} \);
			\item \( \displaystyle L = \left( 4, 9 \right), \quad \frac{-3+x}{2} = \frac{9+y}{11} \);

		\setcounter{tasks}{\value{enumi}}
	\end{enumerate}

	\vspace{10pt}
	\begin{enumerate}
		\setcounter{enumi}{\value{tasks}}

			\item \( \displaystyle X = \left( -1, 7, -2 \right), \quad \frac{x-5}{-8} = \frac{-1+y}{1} = \frac{z-3}{3} \);
			\item \( \displaystyle F = \left( 4, 1, -2 \right), \quad \frac{x-2}{8} = \frac{-6+y}{-15} = \frac{z+3}{-5} \);
			\item \( \displaystyle K = \left( 3, -7, 9 \right), \quad \frac{x-6}{4} = \frac{2+y}{2} = \frac{z-5}{-1} \);
			\item \( \displaystyle J = \left( 0, -9, 9 \right), \quad \frac{x-5}{-8} = \frac{y+1}{-6} = \frac{z-6}{-7} \);
			\item \( \displaystyle B = \left( 9, -5, 8 \right), \quad \frac{x+1}{-8} = \frac{y+8}{10} = \frac{z-1}{9} \);
			\item \( \displaystyle C = \left( 4, 7, -10 \right), \quad \frac{x-5}{-12} = \frac{y-4}{-10} = \frac{4+z}{12} \);
			\item \( \displaystyle E = \left( 2, 7, -3 \right), \quad \frac{1+x}{-4} = \frac{y-4}{4} = \frac{-10+z}{-12} \);
			\item \( \displaystyle S = \left( 1, 5, -7 \right), \quad \frac{x+1}{1} = \frac{y+9}{8} = \frac{z}{5} \);
			\item \( \displaystyle C = \left( -10, -7, 7 \right), \quad \frac{x+8}{-2} = \frac{y-3}{-11} = \frac{-9+z}{-9} \);
			\item \( \displaystyle P = \left( 2, -9, -10 \right), \quad \frac{x}{-10} = \frac{y+10}{3} = \frac{7+z}{8} \);
			\item \( \displaystyle M = \left( -1, -7, 7 \right), \quad \frac{x-9}{-14} = \frac{-5+y}{3} = \frac{z+8}{18} \);
			\item \( \displaystyle H = \left( -5, 9, -4 \right), \quad \frac{-7+x}{-15} = \frac{9+y}{9} = \frac{z-10}{-10} \);

		\setcounter{tasks}{\value{enumi}}
	\end{enumerate}
	\pagebreak

	%---------------------------------------------------------------------------------------------------------------
	%===============================================================================================================
	%---------------------------------------------------------------------------------------------------------------
	
	\input{parts/Vectors/planes.tex}
	\subsection{Задачи}

	Построить каноническое и параметрическое уравнения плоскости, проходящей через следующие точки:

	\begin{enumerate}
		\setcounter{enumi}{\value{tasks}}

			\item \( Q = \pares{2, 1, -3}, ~ S = \pares{3, 1, 4}, ~ J = \pares{4, -2, 0} \);
			\item \( U = \pares{-3, -3, -2}, ~ H = \pares{3, 4, -4}, ~ Z = \pares{-2, 2, 2} \);
			\item \( U = \pares{2, -1, -2}, ~ E = \pares{-3, 4, 4}, ~ L = \pares{-2, 3, -3} \);
			\item \( M = \pares{4, -4, 2}, ~ F = \pares{2, 1, 2}, ~ I = \pares{1, 2, 1} \);
			\item \( R = \pares{0, -1, 1}, ~ G = \pares{4, 4, 3}, ~ X = \pares{-3, -4, -4} \);
			\item \( W = \pares{-4, -2, 3}, ~ C = \pares{1, -4, 4}, ~ R = \pares{2, -2, -3} \);
			\item \( C = \pares{3, 3, -1}, ~ L = \pares{-2, 1, -4}, ~ O = \pares{-4, -1, 3} \);
			\item \( C = \pares{-1, 2, 2}, ~ P = \pares{-4, -2, 3}, ~ Q = \pares{0, -4, 3} \);
			\item \( M = \pares{0, 4, 1}, ~ L = \pares{4, -2, 0}, ~ P = \pares{-2, -1, 2} \);
			\item \( P = \pares{0, -1, 3}, ~ J = \pares{-2, -4, -4}, ~ R = \pares{0, -3, -3} \);
			\item \( J = \pares{2, -4, 2}, ~ K = \pares{2, -4, 4}, ~ P = \pares{2, -4, 4} \);
			\item \( B = \pares{1, 3, 3}, ~ K = \pares{-4, 0, -2}, ~ Y = \pares{-2, -4, -3} \);
			\item \( S = \pares{-4, -1, 2}, ~ B = \pares{-2, -3, 3}, ~ L = \pares{1, 1, 1} \);
			\item \( R = \pares{2, 2, 4}, ~ P = \pares{-4, -3, 2}, ~ S = \pares{3, -2, -3} \);
			\item \( J = \pares{2, 4, 1}, ~ T = \pares{3, 0, -1}, ~ S = \pares{3, 0, -2} \);
			\item \( O = \pares{-3, 4, -4}, ~ D = \pares{-1, 1, 2}, ~ W = \pares{0, 4, 4} \);
			\item \( R = \pares{-2, 2, -2}, ~ C = \pares{-1, -3, -4}, ~ I = \pares{-3, 0, 1} \);
			\item \( Q = \pares{2, 4, 0}, ~ K = \pares{-3, -2, 4}, ~ F = \pares{3, -3, 3} \);
			\item \( H = \pares{-4, -1, 0}, ~ C = \pares{2, -4, 1}, ~ O = \pares{0, 1, -1} \);
			\item \( T = \pares{2, -2, 1}, ~ A = \pares{2, -3, 2}, ~ P = \pares{1, -4, -4} \);

		\setcounter{tasks}{\value{enumi}}
	\end{enumerate}

	\vspace{15pt}
	Построить каноническое и параметрическое уравнения плоскости, основанные на следующих векторах:
	
	\begin{enumerate}
		\setcounter{enumi}{\value{tasks}}

			\item \( \displaystyle \va = \bracs{-9, -10, -9}, ~ \vb = \bracs{-3, -4, -9} \);
			\item \( \displaystyle \va = \bracs{8, 9, -4}, ~ \vb = \bracs{-3, 0, 8} \);
			\item \( \displaystyle \va = \bracs{-1, 10, -10}, ~ \vb = \bracs{8, 7, 3} \);
			\item \( \displaystyle \va = \bracs{8, -10, 0}, ~ \vb = \bracs{4, 1, 3} \);
			\item \( \displaystyle \va = \bracs{-5, -3, -9}, ~ \vb = \bracs{-3, 2, -4} \);
			\item \( \displaystyle \va = \bracs{5, 10, 3}, ~ \vb = \bracs{-3, -5, -7} \);
			\item \( \displaystyle \va = \bracs{-8, -1, -10}, ~ \vb = \bracs{0, -2, 6} \);
			\item \( \displaystyle \va = \bracs{3, 6, -9}, ~ \vb = \bracs{0, 1, 9} \);
			\item \( \displaystyle \va = \bracs{-9, -3, 2}, ~ \vb = \bracs{-4, -10, 7} \);
			\item \( \displaystyle \va = \bracs{-9, 9, -7}, ~ \vb = \bracs{-1, -2, -9} \);
			\item \( \displaystyle \va = \bracs{2, 0, 5}, ~ \vb = \bracs{1, -1, -6} \);
			\item \( \displaystyle \va = \bracs{5, -4, 4}, ~ \vb = \bracs{10, 8, -5} \);
			\item \( \displaystyle \va = \bracs{-9, -6, 2}, ~ \vb = \bracs{-1, 4, 7} \);
			\item \( \displaystyle \va = \bracs{-3, 7, 5}, ~ \vb = \bracs{5, 1, -4} \);
			\item \( \displaystyle \va = \bracs{5, 0, 3}, ~ \vb = \bracs{9, 5, -9} \);
			\item \( \displaystyle \va = \bracs{-4, -7, 3}, ~ \vb = \bracs{-4, -2, -10} \);
			\item \( \displaystyle \va = \bracs{-2, 6, 8}, ~ \vb = \bracs{7, 10, 5} \);
			\item \( \displaystyle \va = \bracs{-6, -1, 0}, ~ \vb = \bracs{2, 9, 1} \);
			\item \( \displaystyle \va = \bracs{1, 2, 8}, ~ \vb = \bracs{4, -2, 10} \);
			\item \( \displaystyle \va = \bracs{10, 9, -1}, ~ \vb = \bracs{-9, -5, 2} \);

		\setcounter{tasks}{\value{enumi}}
	\end{enumerate}

	\vspace{15pt}
	Построить уравнение плоскости, проходящее через точки, полученные проецированием точки $P$ на оси системы координат:

	\begin{multicols}{2}
		\begin{enumerate}
			\setcounter{enumi}{\value{tasks}}

				\item \( P = \pares{-8, -6, 2} \);
				\item \( P = \pares{8, 7, -5} \);
				\item \( P = \pares{4, -3, 4} \);
				\item \( P = \pares{-4, -7, -7} \);
				\item \( P = \pares{-9, -8, -2} \);
				\item \( P = \pares{-7, -7, 2} \);
				\item \( P = \pares{-10, -7, 4} \);
				\item \( P = \pares{2, -2, 8} \);
				\item \( P = \pares{6, 10, 10} \);
				\item \( P = \pares{-9, -9, -3} \);
				\item \( P = \pares{3, -7, 3} \);
				\item \( P = \pares{8, 5, 9} \);
				\item \( P = \pares{-10, 6, 8} \);
				\item \( P = \pares{9, -8, 10} \);
				\item \( P = \pares{-2, 9, -7} \);
				\item \( P = \pares{1, 5, 2} \);
				\item \( P = \pares{-6, -6, -10} \);
				\item \( P = \pares{-10, 5, -3} \);
				\item \( P = \pares{3, 7, 1} \);
				\item \( P = \pares{3, 10, 1} \);

			\setcounter{tasks}{\value{enumi}}
		\end{enumerate}
	\end{multicols}

	\pagebreak
	Построить уравнение плоскости, проходящее через точки $A$ и $B$, и перпендикулярную плоскости $p$:

	\begin{enumerate}
		\setcounter{enumi}{\value{tasks}}

			\item \( A = \pares{-4, -10, 2}, ~ B = \pares{8, -9, 5}, \quad p: 4 + y = 0 \);
			\item \( A = \pares{5, 2, -7}, ~ B = \pares{-3, -8, 4}, \quad p: 5 + y + 2z = 0 \);
			\item \( A = \pares{4, -5, 7}, ~ B = \pares{4, -6, 1}, \quad p: 6x - 19z + 11y = 0 \);
			\item \( A = \pares{-7, -10, -10}, ~ B = \pares{8, 4, 3}, \quad p: 3z + y + 4 + x = 0 \);
			\item \( A = \pares{-2, -7, 2}, ~ B = \pares{3, 2, 5}, \quad p: 2z + 1 + 2x - 3y = 0 \);
			\item \( A = \pares{8, 10, -2}, ~ B = \pares{1, -9, 0}, \quad p: 2x + 8y + 3z = 6 \);
			\item \( A = \pares{9, -5, 8}, ~ B = \pares{-5, 2, -6}, \quad p: 2y - 3x - z + 4 = 0 \);
			\item \( A = \pares{3, 2, 1}, ~ B = \pares{7, -8, -3}, \quad p: 6x + 5y + 4z + 22 = 0 \);
			\item \( A = \pares{3, -8, -6}, ~ B = \pares{4, 3, 10}, \quad p: 10x - 7z + 27 + 15y = 0 \);
			\item \( A = \pares{-3, -7, -9}, ~ B = \pares{6, -7, -9}, \quad p: 6z - x - 17y + 26 = 0 \);
			\item \( A = \pares{10, -7, 10}, ~ B = \pares{-8, -6, -1}, \quad p: 7y - z + 28 - 11x = 0 \);
			\item \( A = \pares{-6, -5, 4}, ~ B = \pares{-9, -3, -8}, \quad p: 12y + 38 - 9x - 28z = 0 \);
			\item \( A = \pares{-3, 4, -6}, ~ B = \pares{1, -8, 10}, \quad p: 16y + 30z + 17x + 26 = 0 \);
			\item \( A = \pares{0, 8, -10}, ~ B = \pares{5, 6, -10}, \quad p: 20 + 30z + x - 18y = 0 \);
			\item \( A = \pares{-6, 6, 4}, ~ B = \pares{-7, 0, 0}, \quad p: 16x - 12z + 31 - 17y = 0 \);
			\item \( A = \pares{7, 6, -4}, ~ B = \pares{0, 2, 5}, \quad p: 6y + 30z + 82 - 7x = 0 \);
			\item \( A = \pares{-4, 2, 0}, ~ B = \pares{9, 8, -10}, \quad p: 35z + 12y + 9 - 30x = 0 \);
			\item \( A = \pares{10, -4, -6}, ~ B = \pares{-4, 8, -10}, \quad p: 68 - x + 19z - 2y = 0 \);
			\item \( A = \pares{-1, 6, 2}, ~ B = \pares{0, 9, -9}, \quad p: 28z + 23x + y = 16 \);
			\item \( A = \pares{10, -10, -8}, ~ B = \pares{6, -5, 10}, \quad p: 11y - 9x + 4z = 24 \);

		\setcounter{tasks}{\value{enumi}}
	\end{enumerate}

	\pagebreak
	Вычислить угол между плоскостями, или показать, что они параллельны:

	\begin{enumerate}
		\setcounter{enumi}{\value{tasks}}

			\item \( y = 1, \quad 13y+4z-24x = 75 \);
			\item \( 6x-3y+10-5z = 0, \quad 1-y-x = 0 \);
			\item \( x+2-y = 0, \quad 13+y+4z-4x = 0 \);
			\item \( z+x-2y = 8, \quad 31z-17y+5x = 22 \);
			\item \( 3x+12+2y-2z = 0, \quad 8-3z-y = 0 \);
			\item \( 4x+16-3z-y = 0, \quad 4z-3x+7 = 0 \);
			\item \( 5x+34z+38y = 16, \quad 4z+3x-4y = 0 \);
			\item \( 5x-2z-6y = 10, \quad 16-16x-7z-10y = 0 \);
			\item \( 11x+38+23z+18y = 0, \quad 8+x+3y = 0 \);
			\item \( 6z-y+6x+14 = 0, \quad 20+12y+z+4x = 0 \);
			\item \( 12y+11z+32x+47 = 0, \quad 7z+2x+7y = 1 \);
			\item \( 32-35x+14z+15y = 0, \quad 8x+35z+48y = 20 \);

		\setcounter{tasks}{\value{enumi}}
	\end{enumerate}

	\vspace{10pt}
	\begin{enumerate}
		\setcounter{enumi}{\value{tasks}}

			\item \( \left\lbrace \begin{aligned}
						x &= -2v-1+4t \\
						y &= -5t-3v+4 \\
						z &= -2t+v-2
					\end{aligned} \right., \quad \left\lbrace \begin{aligned}
						x &= -4+7t+4v \\
						y &= -3+3v-t \\
						z &= -7t-3v+3
					\end{aligned} \right. \);
			\item \( \left\lbrace \begin{aligned}
						x &= v \\
						y &= t-v-1 \\
						z &= -4+7t+6v
					\end{aligned} \right., \quad \left\lbrace \begin{aligned}
						x &= -6t+4 \\
						y &= -4+7t+v \\
						z &= -3v+4-5t
					\end{aligned} \right. \);
			\item \( \left\lbrace \begin{aligned}
						x &= -2v+4-5t \\
						y &= -6v+2-5t \\
						z &= -5t+3-6v
					\end{aligned} \right., \quad \left\lbrace \begin{aligned}
						x &= -t-v \\
						y &= -2v+3 \\
						z &= t-4+4v
					\end{aligned} \right. \);
			\item \( \left\lbrace \begin{aligned}
						x &= -1-t \\
						y &= -5v+1 \\
						z &= -4
					\end{aligned} \right., \quad \left\lbrace \begin{aligned}
						x &= 6v-4 \\
						y &= 3v+6t-3 \\
						z &= v+2t-2
					\end{aligned} \right. \);
			\item \( \left\lbrace \begin{aligned}
						x &= 4v-3t \\
						y &= 4t \\
						z &= -t+3-5v
					\end{aligned} \right., \quad \left\lbrace \begin{aligned}
						x &= -3+t+v \\
						y &= 2+2t+v \\
						z &= 6t-3+4v
					\end{aligned} \right. \);
			\item \( \left\lbrace \begin{aligned}
						x &= 4-8t-2v \\
						y &= 6v+7t-3 \\
						z &= -2+4t+5v
					\end{aligned} \right., \quad \left\lbrace \begin{aligned}
						x &= -1+5t+3v \\
						y &= t \\
						z &= 4v-3+6t
					\end{aligned} \right. \);
			\item \( \left\lbrace \begin{aligned}
						x &= -1+2t+4v \\
						y &= 5t-1 \\
						z &= 2t-4v
					\end{aligned} \right., \quad \left\lbrace \begin{aligned}
						x &= -4v-3t+2 \\
						y &= 4-3t \\
						z &= 5v+7t-3
					\end{aligned} \right. \);
			\item \( \left\lbrace \begin{aligned}
						x &= 2-2v+2t \\
						y &= -3+7v \\
						z &= -4v-2t
					\end{aligned} \right., \quad \left\lbrace \begin{aligned}
						x &= 5t-3+7v \\
						y &= 3-v-5t \\
						z &= 4-8v-5t
					\end{aligned} \right. \);

		\setcounter{tasks}{\value{enumi}}
	\end{enumerate}

	\vspace{15pt}
	Найти точку пересечения прямой $l$ и плоскости $p$, или, если такой не существует -- расстояние между ними:

	\begin{enumerate}
		\setcounter{enumi}{\value{tasks}}

			\item \( \displaystyle l: \frac{x-3}{0} = \frac{-9+y}{-5} = \frac{1+z}{6}, \quad p: z = 0 \);
			\item \( \displaystyle l: \frac{3+x}{11} = \frac{y+9}{14} = \frac{7+z}{14}, \quad p: y = 3 \);
			\item \( \displaystyle l: \frac{x+6}{12} = \frac{y}{-1} = \frac{3+z}{0}, \quad p: 40+3z-4x+9y = 0 \);
			\item \( \displaystyle l: \frac{8+x}{3} = \frac{2+y}{8} = \frac{-4+z}{-3}, \quad p: 26-5x+18z-11y = 0 \);
			\item \( \displaystyle l: \frac{-6+x}{-8} = \frac{9+y}{15} = \frac{z+5}{12}, \quad p: 8x+2+5y+2z = 0 \);
			\item \( \displaystyle l: \frac{4+x}{6} = \frac{7+y}{-3} = \frac{-1+z}{3}, \quad p: 2x-5y+11z = 21 \);
			\item \( \displaystyle l: \frac{x+4}{-2} = \frac{3+y}{9} = \frac{-8+z}{-2}, \quad p: 19x-5z-15y+2 = 0 \);
			\item \( \displaystyle l: \frac{x+1}{2} = \frac{y-5}{4} = \frac{z-2}{8}, \quad p: 23z+2y+14x = 2 \);
			\item \( \displaystyle l: \frac{x-7}{3} = \frac{y+3}{-4} = \frac{-5+z}{-4}, \quad p: 6z-y+4 = 0 \);
			\item \( \displaystyle l: \frac{8+x}{5} = \frac{4+y}{-6} = \frac{z-1}{-5}, \quad p: 6-y-2x = 0 \);
			\item \( \displaystyle l: \frac{x+3}{-6} = \frac{-1+y}{-4} = \frac{-5+z}{-13}, \quad p: 4x-5z+20+2y = 0 \);
			\item \( \displaystyle l: \frac{x+1}{6} = \frac{y-3}{-8} = \frac{2+z}{7}, \quad p: 7x-z+10y = 40 \);

		\setcounter{tasks}{\value{enumi}}
	\end{enumerate}

	\vspace{10pt}
	\begin{enumerate}
		\setcounter{enumi}{\value{tasks}}

			\item \( l: \left\lbrace \begin{aligned}
						x &= 8+2t \\
						y &= 2t+2 \\
						z &= t+1
					\end{aligned} \right., \quad p: \left\lbrace \begin{aligned}
						x &= 7v-4+8t \\
						y &= 3-4t-3v \\
						z &= 4-t-3v
					\end{aligned} \right. \);
			\item \( l: \left\lbrace \begin{aligned}
						x &= -6 \\
						y &= t \\
						z &= -3
					\end{aligned} \right., \quad p: \left\lbrace \begin{aligned}
						x &= 3-6t-5v \\
						y &= 2-3v+2t \\
						z &= 6v-3
					\end{aligned} \right. \);
			\item \( l: \left\lbrace \begin{aligned}
						x &= -7t+9 \\
						y &= 6t-5 \\
						z &= -11t+8
					\end{aligned} \right., \quad p: \left\lbrace \begin{aligned}
						x &= -2+3v+2t \\
						y &= 2v+1 \\
						z &= 2-3t
					\end{aligned} \right. \);
			\item \( l: \left\lbrace \begin{aligned}
						x &= 8t-5 \\
						y &= -9t+4 \\
						z &= -1+6t
					\end{aligned} \right., \quad p: \left\lbrace \begin{aligned}
						x &= -2+t \\
						y &= -1-2v+2t \\
						z &= 2+v
					\end{aligned} \right. \);
			\item \( l: \left\lbrace \begin{aligned}
						x &= -3t+4 \\
						y &= -14t+7 \\
						z &= t+5
					\end{aligned} \right., \quad p: \left\lbrace \begin{aligned}
						x &= 4t-2+3v \\
						y &= -3t+1+2v \\
						z &= v+1+3t
					\end{aligned} \right. \);
			\item \( l: \left\lbrace \begin{aligned}
						x &= -5+11t \\
						y &= -2t-1 \\
						z &= -10+3t
					\end{aligned} \right., \quad p: \left\lbrace \begin{aligned}
						x &= -4+4t+2v \\
						y &= -v-2t \\
						z &= t+4v
					\end{aligned} \right. \);
			\item \( l: \left\lbrace \begin{aligned}
						x &= -3-7t \\
						y &= 10-14t \\
						z &= -1+9t
					\end{aligned} \right., \quad p: \left\lbrace \begin{aligned}
						x &= -2+6v+2t \\
						y &= 2v-3 \\
						z &= -3v+2-5t
					\end{aligned} \right. \);
			\item \( l: \left\lbrace \begin{aligned}
						x &= -10+7t \\
						y &= -2-8t \\
						z &= -1-2t
					\end{aligned} \right., \quad p: \left\lbrace \begin{aligned}
						x &= 3-3t-2v \\
						y &= -4v-4t+2 \\
						z &= 4v-2t
					\end{aligned} \right. \);

		\setcounter{tasks}{\value{enumi}}
	\end{enumerate}

	\vspace{15pt}
	Найти расстояние между прямыми в трехмерном пространстве:

	\begin{enumerate}
		\setcounter{enumi}{\value{tasks}}
		
			\item \( \displaystyle \frac{x}{2} = \frac{2+y}{-6} = \frac{-8+z}{-3}, \quad \frac{2+x}{0} = \frac{-1+y}{-6} = \frac{z+1}{10} \);
			\item \( \displaystyle \frac{x+6}{8} = \frac{y+9}{6} = \frac{z-9}{-10}, \quad \frac{x+8}{11} = \frac{-10+y}{-13} = \frac{-1+z}{-2} \);
			\item \( \displaystyle \frac{8+x}{18} = \frac{y-6}{-5} = \frac{z-9}{-15}, \quad \frac{x-7}{-2} = \frac{y-9}{-11} = \frac{z-4}{-3} \);
			\item \( \displaystyle \frac{x}{10} = \frac{y-6}{2} = \frac{z-10}{-10}, \quad \frac{x-2}{6} = \frac{-7+y}{0} = \frac{z-8}{-8} \);
			\item \( \displaystyle \frac{x-5}{2} = \frac{y+4}{-6} = \frac{2+z}{-1}, \quad \frac{x-9}{1} = \frac{1+y}{4} = \frac{z+10}{20} \);
			\item \( \displaystyle \frac{4+x}{12} = \frac{10+y}{10} = \frac{z-10}{-18}, \quad \frac{x+5}{9} = \frac{y+7}{16} = \frac{-10+z}{-8} \);
			\item \( \displaystyle \frac{x}{4} = \frac{8+y}{13} = \frac{z+10}{4}, \quad \frac{x-4}{5} = \frac{y+3}{8} = \frac{z-3}{-4} \);
			\item \( \displaystyle \frac{x+8}{6} = \frac{1+y}{10} = \frac{z+7}{0}, \quad \frac{x+10}{19} = \frac{3+y}{-6} = \frac{-5+z}{4} \);
			\item \( \displaystyle \frac{x+7}{10} = \frac{-10+y}{-12} = \frac{-7+z}{-3}, \quad \frac{2+x}{-2} = \frac{-3+y}{-8} = \frac{3+z}{6} \);
			\item \( \displaystyle \frac{x+7}{0} = \frac{y+10}{16} = \frac{-9+z}{-1}, \quad \frac{9+x}{16} = \frac{-2+y}{-3} = \frac{z+6}{8} \);
		
		\setcounter{tasks}{\value{enumi}}
	\end{enumerate}
	\pagebreak

	%---------------------------------------------------------------------------------------------------------------
	%===============================================================================================================
	%---------------------------------------------------------------------------------------------------------------

	\newpart{Кривые и поверхности второго порядка и их касательные}

	%---------------------------------------------------------------------------------------------------------------
	%===============================================================================================================
	%---------------------------------------------------------------------------------------------------------------
	
	\input{parts/SecondOrder/curves.tex}
	\subsection{Задачи}

	Классифицировать кривые второго порядка, привести к каноническому виду:

	\begin{multicols}{2}
		\begin{enumerate}
			\setcounter{enumi}{\value{tasks}}

				\item \( 12x - y^2 = 0 \);
				\item \( y^2 - 16 y = 6x - 10 \);
				\item \( 4 x - y^2 = 16 y - 40 \);
				\item \( x^2 - \dfrac{y^2}{16} - \dfrac{9 y}{8} = 2x + \dfrac{81}{16} \);
				\item \( \dfrac{x^2}{64} + \dfrac{y^2}{16} + \dfrac{3 y}{4} + \dfrac{3}{2} = \dfrac{x}{8} \);
				\item \( y^2 + 12y + 56 = 10x \);
				\item \( \dfrac{x^2}{16} + \dfrac{y^2}{16} + \dfrac{y}{8} + \dfrac{85}{16} = \dfrac{5 x}{4} \);
				\item \( y^2 + 18 y + 129 = 12 x \);
				\item \( \dfrac{x^2}{25} + \dfrac{y^2}{25} + \dfrac{3}{5} = \dfrac{12 x}{25} + \dfrac{4 y}{25} \);
				\item \( x^2 - 8 x + 15 = \dfrac{y^2}{81} \);
				\item \( y^2 + 8y = 8x + 40 \);
				\item \( x^2 + 4 x + \dfrac{20 y}{9} = \dfrac{y^2}{9} + \dfrac{73}{9} \);
				\item \( \dfrac{x^2}{36} = \dfrac{x}{3} + \dfrac{y^2}{36} + \dfrac{4 y}{9} + \dfrac{16}{9} \);
				\item \( \dfrac{x^2}{36} + \dfrac{y^2}{81} + \dfrac{20}{9} = \dfrac{5 x}{9} + \dfrac{4 y}{27} \);
				\item \( y^2 + 12 y = 10 x + 4 \);
				\item \( x^2 + 10 x + 2 y + 20 = \dfrac{y^2}{4} \);
				\item \( \dfrac{x^2}{9} + \dfrac{y^2}{4} + 5 y + 24 = 0 \);
				\item \( \dfrac{x^2}{36} + \dfrac{y^2}{64} + \dfrac{y}{8} = \dfrac{x}{18} + \dfrac{13}{18} \);
				\item \( 6 x - y^2 + 4 y = 22 \);
				\item \( \dfrac{x^2}{36} + \dfrac{y^2}{9} + \dfrac{8 y}{9} + \dfrac{7}{9} = 0 \);
				\item \( \dfrac{x^2}{4} = \dfrac{y^2}{49} + \dfrac{12 y}{49} + \dfrac{85}{49} \);
				\item \( 12 x + 4 y + 116 = y^2 \);
				\item \( \dfrac{x^2}{16} = \dfrac{y^2}{25} + \dfrac{12 y}{25} + \dfrac{61}{25} \);
				\item \( \dfrac{x^2}{16} + \dfrac{x}{8} = \dfrac{y^2}{4} + \dfrac{y}{2} + \dfrac{19}{16} \);
				\item \( 6 x - y^2 = 6 y - 3 \);
				\item \( \dfrac{x^2}{25} + \dfrac{y^2}{4} + \dfrac{39}{25} = \dfrac{16 x}{25} \);
				\item \( \dfrac{x^2}{64} + \dfrac{5 x}{32} = \dfrac{y^2}{16} - y + \dfrac{295}{64} \);
				\item \( y^2 - 14 x - 4 y + 130 = 0 \);
				\item \( \dfrac{x^2}{4} + y^2 + 10 y + 49 = 5x \);
				\item \( \dfrac{x^2}{36} + \dfrac{y^2}{36} + \dfrac{y}{9} = \dfrac{8}{9} \);

			\setcounter{tasks}{\value{enumi}}
		\end{enumerate}
	\end{multicols}

	\vspace{15pt}
	Дан эллипс. Найти его полуоси, фокусы, эксцентриситет, уравнения директрис:

	\begin{multicols}{2}
		\begin{enumerate}
			\setcounter{enumi}{\value{tasks}}

				\item \( \dfrac{x^2}{4} + x + \dfrac{y^2}{16} - \dfrac{3 y}{4} + \dfrac{9}{4} = 0 \);
				\item \( \dfrac{x^2}{16} + \dfrac{x}{2} + \dfrac{y^2}{9} + \dfrac{10 y}{9} + \dfrac{25}{9} = 0 \);
				\item \( \dfrac{x^2}{9} + \dfrac{8 x}{9} + \dfrac{y^2}{16} + y + \dfrac{43}{9} = 0 \);
				\item \( x^2 + \dfrac{y^2}{49} + \dfrac{2 y}{7} = 0 \);
				\item \( \dfrac{x^2}{49} + \dfrac{12 x}{49} + \dfrac{y^2}{9} - \dfrac{2 y}{3} + \dfrac{36}{49} = 0 \);
				\item \( \dfrac{x^2}{36} - \dfrac{5 x}{9} + \dfrac{y^2}{36} + \dfrac{y}{9} + \dfrac{17}{9} = 0 \);
				\item \( \dfrac{x^2}{64} + \dfrac{3 x}{16} + \dfrac{y^2}{16} - \dfrac{9 y}{8} + \dfrac{37}{8} = 0 \);
				\item \( \dfrac{x^2}{81} + \dfrac{2 x}{81} + \dfrac{y^2}{9} + \dfrac{2 y}{3} + \dfrac{1}{81} = 0 \);
				\item \( \dfrac{x^2}{9} - \dfrac{2 x}{9} + \dfrac{y^2}{16} - \dfrac{3 y}{4} + \dfrac{49}{36} = 0 \);
				\item \( \dfrac{x^2}{4} - 2 x + \dfrac{y^2}{25} + 3 = 0 \);
				\item \( \dfrac{x^2}{81} - \dfrac{8 x}{81} + \dfrac{y^2}{4} - \dfrac{65}{81} = 0 \);
				\item \( \dfrac{x^2}{16} - \dfrac{5 x}{4} + \dfrac{y^2}{36} - \dfrac{y}{3} + \dfrac{25}{4} = 0 \);
				\item \( x^2 - 20 x + y^2 - 14 y + 148 = 0 \);
				\item \( \dfrac{x^2}{4} + \dfrac{y^2}{9} - \dfrac{16 y}{9} + \dfrac{55}{9} = 0 \);
				\item \( \dfrac{x^2}{4} - \dfrac{5 x}{2} + \dfrac{y^2}{4} - \dfrac{y}{2} + \dfrac{11}{2} = 0 \);
				\item \( x^2 - 10 x + \dfrac{y^2}{81} + \dfrac{2 y}{9} + 25 = 0 \);

			\setcounter{tasks}{\value{enumi}}
		\end{enumerate}
	\end{multicols}

	\pagebreak
	Дана гипербола. Найти: полуоси $a$ и $b$, фокусы, эксцентриситет, уравнения асимптот, уравнения директрис:

	\begin{multicols}{2}
		\begin{enumerate}
			\setcounter{enumi}{\value{tasks}}

				\item \( \dfrac{x^2}{49} - \dfrac{y^2}{36} - \dfrac{y}{2} - \dfrac{13}{4} = 0 \);
				\item \( x^2 + 20 x - \dfrac{y^2}{9} - \dfrac{4 y}{3} + 95 = 0 \);
				\item \( \dfrac{x^2}{4} - \dfrac{9 x}{2} - \dfrac{y^2}{9} - \dfrac{2 y}{3} + \dfrac{73}{4} = 0 \);
				\item \( \dfrac{x^2}{9} - \dfrac{4 x}{3} - \dfrac{y^2}{4} + 2 y - 1 = 0 \);
				\item \( \dfrac{x^2}{16} + \dfrac{7 x}{8} - \dfrac{y^2}{4} + \dfrac{3 y}{2} - \dfrac{3}{16} = 0 \);
				\item \( \dfrac{x^2}{9} - \dfrac{4 x}{3} - \dfrac{y^2}{49} - \dfrac{2 y}{49} + \dfrac{146}{49} = 0 \);
				\item \( \dfrac{x^2}{4} + 4 x - \dfrac{y^2}{81} + \dfrac{2 y}{27} + \dfrac{134}{9} = 0 \);
				\item \( \dfrac{x^2}{4} - 5 x - \dfrac{y^2}{4} + \dfrac{3 y}{2} + \dfrac{87}{4} = 0 \);
				\item \( \dfrac{x^2}{16} + x - y^2 + 12 y - 33 = 0 \);
				\item \( x^2 - 4 x - \dfrac{y^2}{25} + \dfrac{12 y}{25} + \dfrac{39}{25} = 0 \);
				\item \( \dfrac{x^2}{16} + \dfrac{x}{2} - y^2 - 14 y - 49 = 0 \);
				\item \( \dfrac{x^2}{64} - \dfrac{3 x}{16} - \dfrac{y^2}{4} + \dfrac{5 y}{2} - \dfrac{107}{16} = 0 \);
				\item \( \dfrac{x^2}{49} - \dfrac{8 x}{49} - \dfrac{y^2}{49} - \dfrac{20 y}{49} - \dfrac{19}{7} = 0 \);
				\item \( \dfrac{x^2}{25} + \dfrac{6 x}{25} - \dfrac{y^2}{16} - y - \dfrac{116}{25} = 0 \);

			\setcounter{tasks}{\value{enumi}}
		\end{enumerate}
	\end{multicols}

	\vspace{15pt}
	Найти фокус $F$ и уравнение директрисы парабол:
	
	\begin{multicols}{2}
		\begin{enumerate}
			\setcounter{enumi}{\value{tasks}}

			\item \( y^2 - 18 y - 14 x = 31 \);
			\item \( y^2 + 12 y = 8x + 28 \);
			\item \( y^2 = 18 x + 14 y + 59 \);
			\item \( y^2 + 18 y + 73 = 2 x \);
			\item \( y^2 = 6 x + 2 y + 59 \);
			\item \( y^2 + 81 = 8 x + 10 y \);
			\item \( y^2 = 2 \pares{x - y} + 19 \);
			\item \( y^2 = 16 x + 8 y \);
			\item \( y^2 + 20 y + 72 = 4 x \);
			\item \( y^2 + 4 y = 16 x + 12 \);
			\item \( y^2 + 56 = 8 x \);
			\item \( y^2 + 61 = 2 x + 18 y \);
			\item \( y^2 - 6 y + 39 = 10 x \);
			\item \( y^2 + 7 = 2 \pares{x + y} \);
			\item \( y^2 - 18 y + 81 = 16 x \);
			\item \( y^2 + 2 y = 79 + 8 x \);
			\item \( y^2 + 6 y + 1 = 8 x \);
			\item \( y^2 = 14 \pares{x - y} + 91 \);
			\item \( y^2 + 6 y + 37 = 4 x \);
			\item \( y^2 = 4 \pares{x - y} \);

			\setcounter{tasks}{\value{enumi}}
		\end{enumerate}
	\end{multicols}
	
	\vspace{15pt}
	Определить точки пересечения прямой и кривой второго порядка:

	%Парабола
	\begin{enumerate}
		\setcounter{enumi}{\value{tasks}}

			\item \( \displaystyle \frac{-8+x}{-4} = \frac{10+y}{18}, \quad y^2 + 12 y + 12 = 8 x \);
			\item \( \displaystyle \frac{x+7}{4} = \frac{-4+y}{-9}, \quad y^2 + 16 = 4 \pares{x + y} \);
			\item \( \displaystyle \frac{6+x}{9} = \frac{4+y}{-3}, \quad y^2 + 8 y + 112 = 12 x \);
			\item \( \displaystyle \frac{x+9}{3} = \frac{y+6}{-4}, \quad y^2 + 61 = 6 x + 14 y \);
			\item \( \displaystyle \frac{-3+x}{2} = \frac{y}{1}, \quad y^2 + 100 = 12 x + 4 y \);
			\item \( \displaystyle \frac{x+2}{-2} = \frac{1+y}{-5}, \quad y^2 + 4 y = 6 x + 26 \);
			\item \( \displaystyle \frac{-1+x}{-1} = \frac{4+y}{-2}, \quad y^2 + 65 = 2 x + 14 y \);
			\item \( \displaystyle \frac{x+9}{4} = \frac{y+2}{-3}, \quad y^2 + 36 = 4 \pares{x + y} \);
			\item \( \displaystyle \frac{10+x}{8} = \frac{-2+y}{3}, \quad y^2 + 81 = 10 x + 18 y \);
			\item \( \displaystyle \frac{3+x}{-5} = \frac{9+y}{13}, \quad y^2 + 20 y + 112 = 2 x \);
			\item \( \displaystyle \frac{-1+x}{6} = \frac{y+8}{7}, \quad y^2 + 97 = 8 x + 10 y \);
			\item \( \displaystyle \frac{x+9}{8} = \frac{y+9}{5}, \quad y^2 + 14 y + 13 = 12 x \);
			\item \( \displaystyle \frac{x+3}{2} = \frac{y+7}{5}, \quad y^2 = 16 x + 96 \);
			\item \( \displaystyle \frac{x+4}{9} = \frac{y}{-5}, \quad y^2 = 2x + 4 \);
			\item \( \displaystyle \frac{x+1}{-9} = \frac{y-1}{7}, \quad y^2 + 65 = 4 x + 10 y \);
			\item \( \displaystyle \frac{-2+x}{-10} = \frac{-5+y}{3}, \quad y^2 + 20 y + 116 = 8 x \);
			\item \( \displaystyle \frac{x+8}{4} = \frac{y-2}{7}, \quad y^2 + 12 y = 18 \pares{x - 1} \);
			\item \( \displaystyle \frac{-9+x}{0} = \frac{y+10}{17}, \quad y^2 + 225 = 18 \pares{x - y} \);
			\item \( \displaystyle \frac{-1+x}{0} = \frac{8+y}{8}, \quad y^2 = 8 x + 10 y + 39 \);
			\item \( \displaystyle \frac{8+x}{0} = \frac{y}{-5}, \quad y^2 + 22 = 6 x + 4 y \);

		\setcounter{tasks}{\value{enumi}}
	\end{enumerate}

	\vspace{10pt}

	%Эллипс
	\begin{enumerate}
		\setcounter{enumi}{\value{tasks}}

			\item \( \displaystyle \frac{5+x}{7} = \frac{4+y}{7}, \quad \frac{x^2}{4} + \frac{x}{2} + \frac{y^2}{36} - \frac{3}{4} = 0 \);
			\item \( \displaystyle \frac{x-8}{-15} = \frac{-5+y}{-13}, \quad \frac{x^2}{64} + \frac{y^2}{49} - \frac{16 y}{49} + \frac{15}{49} = 0 \);
			\item \( \displaystyle \frac{x-9}{-19} = \frac{y-3}{4}, \quad \frac{x^2}{9} - \frac{8 x}{9} + \frac{y^2}{25} - \frac{2 y}{5} + \frac{16}{9} = 0 \);
			\item \( \displaystyle \frac{x+8}{12} = \frac{y+3}{11}, \quad \frac{x^2}{81} + \frac{4 x}{81} + y^2 + 2 y + \frac{4}{81} = 0 \);
			\item \( \displaystyle \frac{5+x}{13} = \frac{y+1}{-9}, \quad x^2 - 6 x + y^2 + 18 y + 89 = 0 \);
			\item \( \displaystyle \frac{x+9}{-1} = \frac{y-5}{-7}, \quad \frac{x^2}{81} + \frac{y^2}{4} - \frac{9 y}{2} + \frac{77}{4} = 0 \);
			\item \( \displaystyle \frac{5+x}{7} = \frac{6+y}{14}, \quad x^2 - 8 x + \frac{y^2}{9} + 2 y + 24 = 0 \);
			\item \( \displaystyle \frac{x-1}{4} = \frac{-3+y}{-2}, \quad \frac{x^2}{25} - \frac{4 x}{5} + \frac{y^2}{25} + \frac{2 y}{5} + 4 = 0 \);
			\item \( \displaystyle \frac{x-7}{-13} = \frac{2+y}{1}, \quad \frac{x^2}{16} - \frac{5 x}{8} + \frac{y^2}{16} + \frac{y}{2} + \frac{25}{16} = 0 \);
			\item \( \displaystyle \frac{-3+x}{0} = \frac{y}{7}, \quad \frac{x^2}{36} + \frac{5 x}{9} + \frac{y^2}{81} + \frac{4 y}{27} + \frac{20}{9} = 0 \);
			\item \( \displaystyle \frac{x}{5} = \frac{y-2}{-10}, \quad \frac{x^2}{49} + \frac{2 x}{7} + \frac{y^2}{25} + \frac{14 y}{25} + \frac{49}{25} = 0 \);
			\item \( \displaystyle \frac{3+x}{-1} = \frac{y+3}{6}, \quad \frac{x^2}{81} - \frac{2 x}{9} + \frac{y^2}{25} + \frac{14 y}{25} + \frac{49}{25} = 0 \);
			\item \( \displaystyle \frac{x-9}{-12} = \frac{y-5}{-5}, \quad \frac{x^2}{4} + x + \frac{y^2}{36} + \frac{7 y}{18} + \frac{49}{36} = 0 \);
			\item \( \displaystyle \frac{x+7}{13} = \frac{y}{-7}, \quad \frac{x^2}{25} + \frac{2 x}{5} + \frac{y^2}{64} + \frac{9 y}{32} + \frac{81}{64} = 0 \);
			\item \( \displaystyle \frac{x+2}{-3} = \frac{y-9}{0}, \quad \frac{x^2}{9} - \frac{10 x}{9} + \frac{y^2}{9} - \frac{4 y}{9} + \frac{20}{9} = 0 \);
			\item \( \displaystyle \frac{x-9}{-5} = \frac{y-3}{2}, \quad \frac{x^2}{49} - \frac{2 x}{7} + \frac{y^2}{36} - \frac{y}{3} + 1 = 0 \);
			\item \( \displaystyle \frac{-7+x}{-4} = \frac{-7+y}{-10}, \quad \frac{x^2}{64} + \frac{y^2}{81} + \frac{2 y}{81} - \frac{80}{81} = 0 \);
			\item \( \displaystyle \frac{1+x}{8} = \frac{-7+y}{-4}, \quad \frac{x^2}{36} - \frac{x}{3} + \frac{y^2}{9} - \frac{10 y}{9} + \frac{25}{9} = 0 \);
			\item \( \displaystyle \frac{x+10}{3} = \frac{y+5}{-1}, \quad \frac{x^2}{64} + \frac{5 x}{16} + \frac{y^2}{16} - \frac{3 y}{8} + \frac{9}{8} = 0 \);
			\item \( \displaystyle \frac{-3+x}{-7} = \frac{-5+y}{-5}, \quad \frac{x^2}{81} + \frac{20 x}{81} + \frac{y^2}{9} + \frac{2 y}{9} + \frac{28}{81} = 0 \);

		\setcounter{tasks}{\value{enumi}}
	\end{enumerate}

	\vspace{10pt}

	%Гипербола
	\begin{enumerate}
		\setcounter{enumi}{\value{tasks}}

			\item \( \displaystyle \frac{x+1}{-3} = \frac{-7+y}{1}, \quad \frac{x^2}{36} + \frac{5 x}{9} - \frac{y^2}{4} - 3 y - \frac{65}{9} = 0 \);
			\item \( \displaystyle \frac{x+5}{1} = \frac{3+y}{5}, \quad \frac{x^2}{36} + \frac{5 x}{18} - \frac{y^2}{36} - \frac{y}{9} - \frac{5}{12} = 0 \);
			\item \( \displaystyle \frac{-6+x}{-11} = \frac{-3+y}{-1}, \quad \frac{x^2}{9} + \frac{14 x}{9} - \frac{y^2}{9} + \frac{14 y}{9} - 1 = 0 \);
			\item \( \displaystyle \frac{x+6}{4} = \frac{9+y}{7}, \quad \frac{x^2}{81} - \frac{10 x}{81} - \frac{y^2}{36} + \frac{y}{9} - \frac{65}{81} = 0 \);
			\item \( \displaystyle \frac{2+x}{3} = \frac{y-8}{-11}, \quad \frac{x^2}{64} - \frac{3 x}{32} - \frac{y^2}{64} + \frac{y}{8} - \frac{71}{64} = 0 \);
			\item \( \displaystyle \frac{10+x}{16} = \frac{y-6}{3}, \quad \frac{x^2}{36} - \frac{x}{3} - \frac{y^2}{4} - y - 1 = 0 \);
			\item \( \displaystyle \frac{-9+x}{-18} = \frac{y+9}{18}, \quad \frac{x^2}{4} + 5 x - y^2 - 10 y - 1 = 0 \);
			\item \( \displaystyle \frac{9+x}{-1} = \frac{y-7}{-7}, \quad \frac{x^2}{49} + \frac{6 x}{49} - \frac{y^2}{36} - \frac{y}{3} - \frac{89}{49} = 0 \);
			\item \( \displaystyle \frac{x-1}{-8} = \frac{-7+y}{-12}, \quad \frac{x^2}{81} - \frac{2 x}{27} - \frac{y^2}{36} - \frac{y}{18} - \frac{11}{12} = 0 \);
			\item \( \displaystyle \frac{8+x}{0} = \frac{y-5}{-9}, \quad x^2 - 2 x - \frac{y^2}{64} - \frac{3 y}{32} - \frac{9}{64} = 0 \);
			\item \( \displaystyle \frac{3+x}{8} = \frac{y+10}{5}, \quad \frac{x^2}{64} - \frac{x}{8} - \frac{y^2}{9} - \frac{2 y}{3} - \frac{7}{4} = 0 \);
			\item \( \displaystyle \frac{x+4}{6} = \frac{y+1}{5}, \quad \frac{x^2}{49} - \frac{2 x}{7} - \frac{y^2}{16} + \frac{y}{2} - 1 = 0 \);
			\item \( \displaystyle \frac{1+x}{-5} = \frac{-6+y}{-4}, \quad \frac{x^2}{64} - \frac{x}{4} - y^2 + 2 y - 1 = 0 \);
			\item \( \displaystyle \frac{x+1}{5} = \frac{-10+y}{-20}, \quad \frac{x^2}{9} - \frac{2 x}{9} - y^2 - 6 y - \frac{89}{9} = 0 \);
			\item \( \displaystyle \frac{x+8}{11} = \frac{y+6}{2}, \quad \frac{x^2}{64} - \frac{y^2}{81} + \frac{2 y}{27} - \frac{10}{9} = 0 \);
			\item \( \displaystyle \frac{-1+x}{1} = \frac{y-9}{-10}, \quad \frac{x^2}{64} - \frac{x}{16} - y^2 + 4 y - \frac{79}{16} = 0 \);
			\item \( \displaystyle \frac{x+5}{-1} = \frac{y}{-10}, \quad \frac{x^2}{49} + \frac{10 x}{49} - \frac{y^2}{9} - \frac{24}{49} = 0 \);
			\item \( \displaystyle \frac{5+x}{-1} = \frac{y+10}{15}, \quad \frac{x^2}{36} + \frac{x}{6} - \frac{y^2}{16} - \frac{7 y}{8} - \frac{61}{16} = 0 \);
			\item \( \displaystyle \frac{x+8}{10} = \frac{y+4}{-5}, \quad \frac{x^2}{16} - \frac{3 x}{8} - \frac{y^2}{36} - \frac{y}{3} - \frac{23}{16} = 0 \);
			\item \( \displaystyle \frac{x+1}{-8} = \frac{y-3}{-6}, \quad \frac{x^2}{36} + \frac{4 x}{9} - \frac{y^2}{36} - \frac{y}{18} + \frac{3}{4} = 0 \);

		\setcounter{tasks}{\value{enumi}}
	\end{enumerate}
	\pagebreak

	% \noindent Найдите такую функцию $f \in C^1(\mathbb{R})$ (полагая $f \neq 0$), для которой площадь фигуры под графиком модуля этой функции в точности равна периметру фигуры, ограничивающей эту площадь на нефиксированном отрезке $[a, x]$ для любых значений $a < x$; $a, x \in \mathbb{R}$, или покажите, что такой функции не существует.

\end{document}
