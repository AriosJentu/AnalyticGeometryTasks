\subsection{Задачи}

	Построить каноническое и параметрическое уравнения прямой, проходящей через точки в двумерном пространстве:
	\begin{multicols}{2}
		\begin{enumerate}
			\setcounter{enumi}{\value{tasks}}
				
				\item \( \displaystyle B = \pares{1, -1}, ~ N = \pares{-4, 2} \);
				\item \( \displaystyle R = \pares{-8, -2}, ~ I = \pares{8, -7} \);
				\item \( \displaystyle P = \pares{6, -7}, ~ C = \pares{10, -6} \);
				\item \( \displaystyle S = \pares{0, 10}, ~ M = \pares{4, 10} \);
				\item \( \displaystyle F = \pares{8, 9}, ~ L = \pares{-10, -7} \);
				\item \( \displaystyle V = \pares{-5, 8}, ~ E = \pares{-8, -4} \);
				\item \( \displaystyle L = \pares{-9, 1}, ~ F = \pares{-5, -6} \);
				\item \( \displaystyle H = \pares{3, 9}, ~ G = \pares{-1, -1} \);
				\item \( \displaystyle A = \pares{7, 5}, ~ O = \pares{-4, -2} \);
				\item \( \displaystyle H = \pares{-4, 7}, ~ R = \pares{-9, -1} \);
				\item \( \displaystyle Q = \pares{6, 0}, ~ K = \pares{7, -8} \);
				\item \( \displaystyle D = \pares{-4, 9}, ~ Y = \pares{-5, -2} \);
				\item \( \displaystyle J = \pares{-1, 0}, ~ S = \pares{-8, -5} \);
				\item \( \displaystyle Y = \pares{5, 5}, ~ Y = \pares{-7, -4} \);
				\item \( \displaystyle A = \pares{-2, 0}, ~ X = \pares{6, -9} \);
				\item \( \displaystyle J = \pares{5, 4}, ~ K = \pares{1, -5} \);
				\item \( \displaystyle G = \pares{-6, 4}, ~ H = \pares{5, -5} \);
				\item \( \displaystyle S = \pares{6, 9}, ~ N = \pares{4, 8} \);
				\item \( \displaystyle G = \pares{-6, -8}, ~ F = \pares{-8, 7} \);
				\item \( \displaystyle Q = \pares{7, 6}, ~ W = \pares{8, -3} \);

			\setcounter{tasks}{\value{enumi}}
		\end{enumerate}
	\end{multicols}

	\vspace{15pt}
	Построить параметрическое уравнение для следующих прямых:

	\begin{multicols}{2}
		\begin{enumerate}
			\setcounter{enumi}{\value{tasks}}
				
				\item \( \displaystyle y = \frac{6+x}{-4} \);
				\item \( \displaystyle x + 3 = \frac{-5+y}{-4} \);
				\item \( \displaystyle \frac{7+x}{6} = \frac{y+9}{16} \);
				\item \( \displaystyle \frac{6+x}{-1} = \frac{y-6}{-5} \);
				\item \( \displaystyle \frac{x-3}{6} = \frac{y-1}{-6} \);
				\item \( \displaystyle \frac{x+10}{9} = \frac{y+9}{18} \);
				\item \( \displaystyle \frac{-5+x}{-2} = \frac{y+2}{2} \);
				\item \( \displaystyle \frac{x+3}{4} = \frac{y-4}{-8} \);
				\item \( \displaystyle \frac{-4+x}{1} = \frac{-10+y}{-10} \);
				\item \( \displaystyle \frac{10+x}{8} = \frac{y+5}{12} \);
				\item \( \displaystyle \frac{x-2}{-2} = \frac{y-2}{8} \);
				\item \( \displaystyle \frac{x+5}{11} = \frac{6+y}{3} \);

			\setcounter{tasks}{\value{enumi}}
		\end{enumerate}
	\end{multicols}

	\vspace{10pt}
	\begin{multicols}{2}
		\begin{enumerate}
			\setcounter{enumi}{\value{tasks}}

				\item \( \displaystyle x-8 = 9-y = \frac{z-5}{3} \);
				\item \( \displaystyle \frac{x-2}{-10} = \frac{-8+y}{-8} = \frac{z+5}{-3} \);
				\item \( \displaystyle \frac{-1+x}{3} = \frac{9+y}{3} = \frac{z-5}{-1} \);
				\item \( \displaystyle \frac{-3+x}{2} = \frac{y-7}{-17} = \frac{-2+z}{-8} \);
				\item \( \displaystyle \frac{-9+x}{-15} = \frac{3+y}{11} = \frac{z-5}{-10} \);
				\item \( \displaystyle \frac{3+x}{0} = \frac{y+4}{8} = \frac{4+z}{8} \);
				\item \( \displaystyle \frac{x-1}{0} = \frac{-3+y}{4} = \frac{z-10}{-14} \);
				\item \( \displaystyle \frac{x-6}{-4} = \frac{y-4}{-12} = \frac{z+1}{0} \);

			\setcounter{tasks}{\value{enumi}}
		\end{enumerate}
	\end{multicols}

	\vspace{15pt}
	Построить вектор нормали для следующих прямых на плоскости:

	\begin{multicols}{2}
		\begin{enumerate}
			\setcounter{enumi}{\value{tasks}}

				\item \( \displaystyle y = 4 - x \)
				\item \( \displaystyle y = 5 - 8x \)
				\item \( \displaystyle y = 7x + 5 \)
				\item \( \displaystyle y = 4x - 3 \)

			\setcounter{tasks}{\value{enumi}}
		\end{enumerate}
	\end{multicols}

	\vspace{10pt}
	\begin{multicols}{2}
		\begin{enumerate}
			\setcounter{enumi}{\value{tasks}}

				\item \( \displaystyle y = \frac{3}{5} x + \frac{41}{5} \)
				\item \( \displaystyle y = \frac{3}{4} x - \frac{5}{4} \)
				\item \( \displaystyle y = \frac{3}{4} x + \frac{1}{4} \)
				\item \( \displaystyle y = \frac{3}{5} x + \frac{3}{5} \)

			\setcounter{tasks}{\value{enumi}}
		\end{enumerate}
	\end{multicols}

	\vspace{10pt}
	\begin{multicols}{2}
		\begin{enumerate}
			\setcounter{enumi}{\value{tasks}}

				\item \( \displaystyle \frac{9+x}{17} = \frac{y+8}{4} \);
				\item \( \displaystyle \frac{4+x}{13} = \frac{y+7}{14} \);
				\item \( \displaystyle \frac{5+x}{0} = \frac{-4+y}{5} \);
				\item \( \displaystyle \frac{-3+x}{-7} = \frac{-1+y}{0} \);
				\item \( \displaystyle \frac{x-7}{-13} = \frac{-1+y}{2} \);
				\item \( \displaystyle \frac{x-10}{-16} = \frac{y}{2} \);
				\item \( \displaystyle \frac{-8+x}{-12} = \frac{y-7}{1} \);
				\item \( \displaystyle \frac{9+x}{19} = \frac{6+y}{7} \);

			\setcounter{tasks}{\value{enumi}}
		\end{enumerate}
	\end{multicols}

	\vspace{10pt}
	\begin{multicols}{2}
		\begin{enumerate}
			\setcounter{enumi}{\value{tasks}}

				\item \( \displaystyle \left\lbrace \begin{aligned}
							x &= t - 3 \\
							y &= 1
						\end{aligned} \right. \);
				\item \( \displaystyle \left\lbrace \begin{aligned}
							x &= 1 \\
							y &= 4 + t
						\end{aligned} \right. \);
				\item \( \displaystyle \left\lbrace \begin{aligned}
							x &= 9 - 9t \\
							y &= -3 - t
						\end{aligned} \right. \);
				\item \( \displaystyle \left\lbrace \begin{aligned}
							x &= t - 3 \\
							y &= -7 + 5t
						\end{aligned} \right. \);

			\setcounter{tasks}{\value{enumi}}
		\end{enumerate}
	\end{multicols}

	\vspace{15pt}
	Построить каноническое и параметрическое уравнения прямой, проходящей через точку $M$, перпендикулярной вектору $\vn$:

	\begin{multicols}{2}
		\begin{enumerate}
			\setcounter{enumi}{\value{tasks}}
					
				\item \( \displaystyle M = \pares{7, -4}, ~ \vn = \bracs{1, -6} \);
				\item \( \displaystyle M = \pares{-2, 8}, ~ \vn = \bracs{-6, 6} \);
				\item \( \displaystyle M = \pares{4, 10}, ~ \vn = \bracs{-1, -2} \);
				\item \( \displaystyle M = \pares{4, -6}, ~ \vn = \bracs{2, 5} \);
				\item \( \displaystyle M = \pares{-9, 7}, ~ \vn = \bracs{-1, 4} \);
				\item \( \displaystyle M = \pares{-4, 1}, ~ \vn = \bracs{0, -5} \);
				\item \( \displaystyle M = \pares{5, -1}, ~ \vn = \bracs{8, -8} \);
				\item \( \displaystyle M = \pares{2, -2}, ~ \vn = \bracs{-3, 3} \);
				\item \( \displaystyle M = \pares{-9, 6}, ~ \vn = \bracs{-2, 4} \);
				\item \( \displaystyle M = \pares{10, 1}, ~ \vn = \bracs{-1, 5} \);
				\item \( \displaystyle M = \pares{1, 2}, ~ \vn = \bracs{-2, 10} \);
				\item \( \displaystyle M = \pares{-2, 2}, ~ \vn = \bracs{4, 2} \);
				\item \( \displaystyle M = \pares{-9, 7}, ~ \vn = \bracs{8, -4} \);
				\item \( \displaystyle M = \pares{-10, -8}, ~ \vn = \bracs{5, 8} \);
				\item \( \displaystyle M = \pares{-8, 7}, ~ \vn = \bracs{-9, 5} \);
				\item \( \displaystyle M = \pares{-9, 9}, ~ \vn = \bracs{6, 8} \);
				\item \( \displaystyle M = \pares{-7, 6}, ~ \vn = \bracs{3, -6} \);
				\item \( \displaystyle M = \pares{10, -3}, ~ \vn = \bracs{-4, 6} \);
				\item \( \displaystyle M = \pares{9, -5}, ~ \vn = \bracs{-8, 1} \);
				\item \( \displaystyle M = \pares{7, -10}, ~ \vn = \bracs{-2, 2} \);

			\setcounter{tasks}{\value{enumi}}
		\end{enumerate}
	\end{multicols}

	\pagebreak
	Построить каноническое и параметрическое уравнения прямой, проходящей через точку $O$, построенной по направлению вектора $\vv$ в трехмерном пространстве:

	\begin{enumerate}
		\setcounter{enumi}{\value{tasks}}

			\item \( \displaystyle O = \pares{-5, -6, -9}, ~ \vv = \bracs{-1, -3, -10} \);
			\item \( \displaystyle O = \pares{-2, 6, -1}, ~ \vv = \bracs{0, -5, 4} \);
			\item \( \displaystyle O = \pares{9, -2, 6}, ~ \vv = \bracs{-4, 9, 7} \);
			\item \( \displaystyle O = \pares{10, 3, 7}, ~ \vv = \bracs{7, -3, -9} \);
			\item \( \displaystyle O = \pares{4, 4, -10}, ~ \vv = \bracs{4, 4, -9} \);
			\item \( \displaystyle O = \pares{0, 0, 4}, ~ \vv = \bracs{-5, -1, 2} \);
			\item \( \displaystyle O = \pares{1, 10, -9}, ~ \vv = \bracs{4, -10, 8} \);
			\item \( \displaystyle O = \pares{-3, 9, -10}, ~ \vv = \bracs{7, -8, 10} \);
			\item \( \displaystyle O = \pares{-10, -1, 0}, ~ \vv = \bracs{6, 3, 8} \);
			\item \( \displaystyle O = \pares{-4, 0, 3}, ~ \vv = \bracs{4, 10, 5} \);

		\setcounter{tasks}{\value{enumi}}
	\end{enumerate}

	\vspace{15pt}
	Найти направляющие вектора и минимальный угол между прямыми, или показать, что они параллельны:

	\begin{multicols}{2}
		\begin{enumerate}
			\setcounter{enumi}{\value{tasks}}

				\item \( \displaystyle y = 6 - x, ~ y = 2 + 2x \);
				\item \( \displaystyle y = x + 2, ~ y = 2x - 5 \);
				\item \( \displaystyle y = 5x + 4, ~ y = 3x - 4  \);
				\item \( \displaystyle y = 4 - 7x, ~ y = 4x - 7 \);

			\setcounter{tasks}{\value{enumi}}
		\end{enumerate}
	\end{multicols}

	\vspace{10pt}
	\begin{enumerate}
		\setcounter{enumi}{\value{tasks}}

			\item \( \displaystyle \frac{x+6}{2} = \frac{-10+y}{-17}, \quad \frac{6+x}{2} = \frac{y-10}{-17} \);
			\item \( \displaystyle \frac{x-7}{-17} = \frac{8+y}{1}, \quad \frac{-7+x}{-17} = \frac{8+y}{1} \);
			\item \( \displaystyle \frac{x-3}{-4} = \frac{y+8}{-2}, \quad \frac{x-3}{-4} = \frac{y+8}{-2} \);
			\item \( \displaystyle \frac{1+x}{1} = \frac{y-4}{5}, \quad \frac{x+1}{1} = \frac{-4+y}{5} \);
			\item \( \displaystyle \frac{x+4}{-3} = \frac{-6+y}{-2}, \quad \frac{4+x}{-3} = \frac{-6+y}{-2} \);
			\item \( \displaystyle \frac{x+4}{13} = \frac{-6+y}{-6}, \quad \frac{x+4}{13} = \frac{y-6}{-6} \);

		\setcounter{tasks}{\value{enumi}}
	\end{enumerate}

	\vspace{10pt}
	\begin{enumerate}
		\setcounter{enumi}{\value{tasks}}
			\item \( \displaystyle \left\lbrace \begin{aligned}
						x &= 3-2t \\
						y &= 2t-2
					\end{aligned} \right., \left\lbrace \begin{aligned}
						x &= -2t+3 \\
						y &= 2t-2
					\end{aligned} \right. \);
			\item \( \displaystyle \left\lbrace \begin{aligned}
						x &= -8t+3 \\
						y &= 3t-6
					\end{aligned} \right., \left\lbrace \begin{aligned}
						x &= 3-8t \\
						y &= 3t-6
					\end{aligned} \right. \);
			\item \( \displaystyle \left\lbrace \begin{aligned}
						x &= 15t-5 \\
						y &= 7t
					\end{aligned} \right., \left\lbrace \begin{aligned}
						x &= 15t-5 \\
						y &= 7t
					\end{aligned} \right. \);
			\item \( \displaystyle \left\lbrace \begin{aligned}
						x &= -18t+9 \\
						y &= -20t+10
					\end{aligned} \right., \left\lbrace \begin{aligned}
						x &= -18t+9 \\
						y &= 10-20t
					\end{aligned} \right. \);

		\setcounter{tasks}{\value{enumi}}
	\end{enumerate}

	\vspace{10pt}
	\begin{enumerate}
		\setcounter{enumi}{\value{tasks}}

			\item \( \displaystyle \frac{x}{-2} = \frac{-1+y}{4} = \frac{-2+z}{8}, \quad \frac{x}{-2} = \frac{y-1}{4} = \frac{-2+z}{8} \);
			\item \( \displaystyle \frac{x-5}{0} = \frac{4+y}{-2} = \frac{7+z}{3}, \quad \frac{-5+x}{0} = \frac{4+y}{-2} = \frac{z+7}{3} \);
			\item \( \displaystyle \frac{x-5}{1} = \frac{y+9}{19} = \frac{1+z}{3}, \quad \frac{x-5}{1} = \frac{9+y}{19} = \frac{1+z}{3} \);
			\item \( \displaystyle \frac{x+5}{-3} = \frac{-9+y}{-5} = \frac{4+z}{14}, \quad \frac{x+5}{-3} = \frac{y-9}{-5} = \frac{z+4}{14} \);
			\item \( \displaystyle \frac{x}{2} = \frac{-4+y}{4} = \frac{3+z}{8}, \quad \frac{x}{2} = \frac{-4+y}{4} = \frac{3+z}{8} \);
			\item \( \displaystyle \frac{6+x}{4} = \frac{y}{9} = \frac{z-2}{-8}, \quad \frac{6+x}{4} = \frac{y}{9} = \frac{-2+z}{-8} \);
			\item \( \displaystyle \frac{x+10}{6} = \frac{-2+y}{-7} = \frac{z+7}{16}, \quad \frac{x+10}{6} = \frac{y-2}{-7} = \frac{z+7}{16} \);
			\item \( \displaystyle \frac{x}{7} = \frac{3+y}{8} = \frac{-2+z}{8}, \quad \frac{x}{7} = \frac{y+3}{8} = \frac{-2+z}{8} \);
			\item \( \displaystyle \frac{x+2}{-4} = \frac{9+y}{5} = \frac{z+10}{0}, \quad \frac{2+x}{-4} = \frac{y+9}{5} = \frac{z+10}{0} \);
			\item \( \displaystyle \frac{5+x}{-5} = \frac{y+4}{-1} = \frac{z-2}{3}, \quad \frac{5+x}{-5} = \frac{4+y}{-1} = \frac{-2+z}{3} \);
			\item \( \displaystyle \frac{x+1}{3} = \frac{5+y}{-3} = \frac{z+5}{6}, \quad \frac{1+x}{3} = \frac{5+y}{-3} = \frac{z+5}{6} \);
			\item \( \displaystyle \frac{x-6}{-4} = \frac{7+y}{12} = \frac{z-7}{-6}, \quad \frac{x-6}{-4} = \frac{y+7}{12} = \frac{-7+z}{-6} \);
			\item \( \displaystyle \frac{10+x}{12} = \frac{6+y}{16} = \frac{-3+z}{-7}, \quad \frac{x+10}{12} = \frac{y+6}{16} = \frac{z-3}{-7} \);
			\item \( \displaystyle \frac{x-5}{-6} = \frac{y+6}{4} = \frac{-8+z}{-8}, \quad \frac{-5+x}{-6} = \frac{6+y}{4} = \frac{-8+z}{-8} \);
			\item \( \displaystyle \frac{x+1}{3} = \frac{9+y}{8} = \frac{z-1}{9}, \quad \frac{1+x}{3} = \frac{9+y}{8} = \frac{-1+z}{9} \);
			\item \( \displaystyle \frac{4+x}{-5} = \frac{y-6}{1} = \frac{-10+z}{-8}, \quad \frac{x+4}{-5} = \frac{y-6}{1} = \frac{z-10}{-8} \);

		\setcounter{tasks}{\value{enumi}}
	\end{enumerate}

	\vspace{15pt}
	Построить прямую, проходящую через точку $M$, поверную на угол $\alpha$ относительно прямой $l$:

	\begin{enumerate}
		\setcounter{enumi}{\value{tasks}}

			\item \( \displaystyle M = \left( 1, -3 \right), ~ \alpha = -\dfrac{3\pi}{4}, \quad l: \frac{-3+x}{-13} = \frac{y+10}{10} \);
			\item \( \displaystyle M = \left( -1, 1 \right), ~ \alpha = -\dfrac{5\pi}{3}, \quad l: \frac{9+x}{9} = \frac{y+4}{7} \);
			\item \( \displaystyle M = \left( -3, -1 \right), ~ \alpha = \dfrac{7\pi}{4}, \quad l: \frac{-2+x}{6} = \frac{5+y}{15} \);
			\item \( \displaystyle M = \left( 9, -4 \right), ~ \alpha = -\pi, \quad l: \frac{x-5}{-4} = \frac{y+10}{19} \);
			\item \( \displaystyle M = \left( -7, 9 \right), ~ \alpha = \dfrac{7\pi}{4}, \quad l: \frac{x-7}{3} = \frac{y+6}{0} \);
			\item \( \displaystyle M = \left( 3, 9 \right), ~ \alpha = -\dfrac{2\pi}{3}, \quad l: \frac{x+8}{7} = \frac{y+7}{13} \);
			\item \( \displaystyle M = \left( -2, 3 \right), ~ \alpha = \dfrac{7\pi}{6}, \quad l: \frac{x+5}{10} = \frac{y+6}{15} \);
			\item \( \displaystyle M = \left( -5, 6 \right), ~ \alpha = -\dfrac{3\pi}{2}, \quad l: \frac{6+x}{13} = \frac{10+y}{2} \);
			\item \( \displaystyle M = \left( -6, 4 \right), ~ \alpha = \dfrac{7\pi}{6}, \quad l: \frac{x-1}{-10} = \frac{-8+y}{-6} \);
			\item \( \displaystyle M = \left( 5, 3 \right), ~ \alpha = \dfrac{4\pi}{3}, \quad l: \frac{x-5}{-2} = \frac{y+5}{9} \);
			\item \( \displaystyle M = \left( 8, -6 \right), ~ \alpha = \pi, \quad l: \frac{-7+x}{-5} = \frac{y-9}{1} \);
			\item \( \displaystyle M = \left( 0, 1 \right), ~ \alpha = -\dfrac{3\pi}{2}, \quad l: \frac{-8+x}{-14} = \frac{5+y}{10} \);

		\setcounter{tasks}{\value{enumi}}
	\end{enumerate}

	\vspace{15pt}
	Найти расстояние от точки до прямой:

	\begin{enumerate}
		\setcounter{enumi}{\value{tasks}}
			
			\item \( \displaystyle B = \left( 6, 3 \right), \quad \frac{x+8}{18} = \frac{-10+y}{-17} \);
			\item \( \displaystyle J = \left( -8, -8 \right), \quad \frac{-4+x}{6} = \frac{7+y}{8} \);
			\item \( \displaystyle W = \left( -4, 3 \right), \quad \frac{-6+x}{-10} = \frac{y+9}{12} \);
			\item \( \displaystyle O = \left( -2, 10 \right), \quad \frac{-2+x}{-9} = \frac{y+5}{13} \);
			\item \( \displaystyle A = \left( -9, -10 \right), \quad \frac{x-8}{-14} = \frac{10+y}{10} \);
			\item \( \displaystyle B = \left( 4, 1 \right), \quad \frac{x+1}{-9} = \frac{8+y}{14} \);
			\item \( \displaystyle V = \left( 9, 0 \right), \quad \frac{-7+x}{2} = \frac{y+4}{-3} \);
			\item \( \displaystyle Z = \left( 4, -7 \right), \quad \frac{x}{1} = \frac{5+y}{13} \);
			\item \( \displaystyle C = \left( -9, -10 \right), \quad \frac{-5+x}{4} = \frac{y-5}{-13} \);
			\item \( \displaystyle M = \left( -3, 6 \right), \quad \frac{4+x}{12} = \frac{y-5}{2} \);
			\item \( \displaystyle P = \left( -1, 3 \right), \quad \frac{-2+x}{-10} = \frac{1+y}{3} \);
			\item \( \displaystyle V = \left( -3, 1 \right), \quad \frac{7+x}{12} = \frac{-6+y}{-5} \);
			\item \( \displaystyle X = \left( -6, -2 \right), \quad \frac{-10+x}{-16} = \frac{y-4}{-9} \);
			\item \( \displaystyle W = \left( 10, 6 \right), \quad \frac{x-4}{2} = \frac{4+y}{-1} \);
			\item \( \displaystyle G = \left( 8, -5 \right), \quad \frac{x-7}{-16} = \frac{y+9}{4} \);
			\item \( \displaystyle L = \left( 4, 9 \right), \quad \frac{-3+x}{2} = \frac{9+y}{11} \);

		\setcounter{tasks}{\value{enumi}}
	\end{enumerate}

	\vspace{10pt}
	\begin{enumerate}
		\setcounter{enumi}{\value{tasks}}

			\item \( \displaystyle X = \left( -1, 7, -2 \right), \quad \frac{x-5}{-8} = \frac{-1+y}{1} = \frac{z-3}{3} \);
			\item \( \displaystyle F = \left( 4, 1, -2 \right), \quad \frac{x-2}{8} = \frac{-6+y}{-15} = \frac{z+3}{-5} \);
			\item \( \displaystyle K = \left( 3, -7, 9 \right), \quad \frac{x-6}{4} = \frac{2+y}{2} = \frac{z-5}{-1} \);
			\item \( \displaystyle J = \left( 0, -9, 9 \right), \quad \frac{x-5}{-8} = \frac{y+1}{-6} = \frac{z-6}{-7} \);
			\item \( \displaystyle B = \left( 9, -5, 8 \right), \quad \frac{x+1}{-8} = \frac{y+8}{10} = \frac{z-1}{9} \);
			\item \( \displaystyle C = \left( 4, 7, -10 \right), \quad \frac{x-5}{-12} = \frac{y-4}{-10} = \frac{4+z}{12} \);
			\item \( \displaystyle E = \left( 2, 7, -3 \right), \quad \frac{1+x}{-4} = \frac{y-4}{4} = \frac{-10+z}{-12} \);
			\item \( \displaystyle S = \left( 1, 5, -7 \right), \quad \frac{x+1}{1} = \frac{y+9}{8} = \frac{z}{5} \);
			\item \( \displaystyle C = \left( -10, -7, 7 \right), \quad \frac{x+8}{-2} = \frac{y-3}{-11} = \frac{-9+z}{-9} \);
			\item \( \displaystyle P = \left( 2, -9, -10 \right), \quad \frac{x}{-10} = \frac{y+10}{3} = \frac{7+z}{8} \);
			\item \( \displaystyle M = \left( -1, -7, 7 \right), \quad \frac{x-9}{-14} = \frac{-5+y}{3} = \frac{z+8}{18} \);
			\item \( \displaystyle H = \left( -5, 9, -4 \right), \quad \frac{-7+x}{-15} = \frac{9+y}{9} = \frac{z-10}{-10} \);

		\setcounter{tasks}{\value{enumi}}
	\end{enumerate}