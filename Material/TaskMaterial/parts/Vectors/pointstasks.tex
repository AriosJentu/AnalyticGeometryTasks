\subsection{Задачи}

	Найти вектор перехода от точки $A$ к точке $B$:
	\begin{enumerate}
		\setcounter{enumi}{\value{tasks}}
			
			\item \( \displaystyle A = \pares{2, 10, 2}, ~ B = \pares{6, 3, 5} \);
			\item \( \displaystyle A = \pares{-6, 9, 1}, ~ B = \pares{1, -3, 2} \);
			\item \( \displaystyle A = \pares{6, 6, 5}, ~ B = \pares{-2, 8, -10} \);
			\item \( \displaystyle A = \pares{6, 1, -3}, ~ B = \pares{2, -4, 5} \);
			\item \( \displaystyle A = \pares{-3, 2, 5}, ~ B = \pares{10, 1, 0} \);
			\item \( \displaystyle A = \pares{-4, 8, -7}, ~ B = \pares{0, 6, -1} \);
			\item \( \displaystyle A = \pares{-2, -8, -1}, ~ B = \pares{-7, -6, -1} \);
			\item \( \displaystyle A = \pares{-4, -2, 7}, ~ B = \pares{-8, -2, 7} \);
			\item \( \displaystyle A = \pares{4, -3, -5}, ~ B = \pares{-7, -2, 2} \);
			\item \( \displaystyle A = \pares{9, -3, -3}, ~ B = \pares{-9, 9, 4} \);
			\item \( \displaystyle A = \pares{-10, 10, 2}, ~ B = \pares{-7, -1, -8} \);
			\item \( \displaystyle A = \pares{-2, 10, -4}, ~ B = \pares{-5, 6, 0} \);
			\item \( \displaystyle A = \pares{3, 0, -8}, ~ B = \pares{6, 0, 8} \);
			\item \( \displaystyle A = \pares{6, -7, -8}, ~ B = \pares{-9, 7, 4} \);
			\item \( \displaystyle A = \pares{-8, 0, -10}, ~ B = \pares{4, 7, 0} \);
			\item \( \displaystyle A = \pares{-9, 7, 4}, ~ B = \pares{4, -9, -1} \);
			\item \( \displaystyle A = \pares{-7, -10, -9}, ~ B = \pares{10, 3, 10} \);
			\item \( \displaystyle A = \pares{-4, 1, -8}, ~ B = \pares{-1, 2, -5} \);
			\item \( \displaystyle A = \pares{-5, -6, -8}, ~ B = \pares{8, 5, 6} \);
			\item \( \displaystyle A = \pares{1, 5, 8}, ~ B = \pares{4, -3, 3} \);

		\setcounter{tasks}{\value{enumi}}
	\end{enumerate}

	\vspace{15pt}
	Найти в двумерном пространстве положение точки $C$, полученной делением отрезка $AB$ в соотношении $\lambda = \dfrac{AC}{CB}$:

	\begin{enumerate}
		\setcounter{enumi}{\value{tasks}}

		\item \( \displaystyle A = \pares{2, 7}, ~ B = \pares{1, -10}, ~ \lambda = \frac{1}{2} \);
		\item \( \displaystyle A = \pares{-1, 1}, ~ B = \pares{6, -9}, ~ \lambda = \frac{1}{3} \);
		\item \( \displaystyle A = \pares{9, 2}, ~ B = \pares{-10, -9}, ~ \lambda = \frac{1}{3} \);
		\item \( \displaystyle A = \pares{0, 3}, ~ B = \pares{4, -9}, ~ \lambda = \frac{1}{3} \);
		\item \( \displaystyle A = \pares{-8, -6}, ~ B = \pares{2, -1}, ~ \lambda = \frac{2}{3} \);
		\item \( \displaystyle A = \pares{7, -3}, ~ B = \pares{-1, -4}, ~ \lambda = \frac{2}{3} \);
		\item \( \displaystyle A = \pares{-3, -1}, ~ B = \pares{7, 6}, ~ \lambda = \frac{1}{4} \);
		\item \( \displaystyle A = \pares{-9, 9}, ~ B = \pares{0, -1}, ~ \lambda = \frac{3}{4} \);
		\item \( \displaystyle A = \pares{-7, -10}, ~ B = \pares{-7, 0}, ~ \lambda = \frac{3}{4} \);
		\item \( \displaystyle A = \pares{8, 0}, ~ B = \pares{0, -5}, ~ \lambda = \frac{1}{5} \);
		\item \( \displaystyle A = \pares{-7, 0}, ~ B = \pares{9, 8}, ~ \lambda = \frac{2}{5} \);
		\item \( \displaystyle A = \pares{1, 5}, ~ B = \pares{2, 2}, ~ \lambda = \frac{1}{6} \);
		\item \( \displaystyle A = \pares{0, 3}, ~ B = \pares{4, 6}, ~ \lambda = \frac{5}{6} \);
		\item \( \displaystyle A = \pares{-6, 8}, ~ B = \pares{1, 3}, ~ \lambda = \frac{3}{7} \);
		\item \( \displaystyle A = \pares{6, -7}, ~ B = \pares{-1, -2}, ~ \lambda = \frac{4}{7} \);
		\item \( \displaystyle A = \pares{7, 7}, ~ B = \pares{-3, 6}, ~ \lambda = \frac{5}{7} \);
		\item \( \displaystyle A = \pares{7, -2}, ~ B = \pares{4, -3}, ~ \lambda = \frac{5}{8} \);
		\item \( \displaystyle A = \pares{-4, -5}, ~ B = \pares{-3, 8}, ~ \lambda = \frac{4}{9} \);

		\setcounter{tasks}{\value{enumi}}
	\end{enumerate}

	\vspace{15pt}
	Найти в трехмерном пространстве положение точки $C$, полученной делением отрезка $AB$ в соотношении $\lambda = \dfrac{AC}{CB}$:
	\begin{enumerate}
		\setcounter{enumi}{\value{tasks}}

		\item \( \displaystyle A = \pares{-6, -1, 7}, ~ B = \pares{-7, -8, 9}, ~ \lambda = \frac{1}{2} \);
		\item \( \displaystyle A = \pares{-1, -7, 0}, ~ B = \pares{-2, -5, 5}, ~ \lambda = \frac{1}{2} \);
		\item \( \displaystyle A = \pares{2, 0, -7}, ~ B = \pares{6, -10, 10}, ~ \lambda = \frac{1}{3} \);
		\item \( \displaystyle A = \pares{-6, 10, -5}, ~ B = \pares{10, -5, -3}, ~ \lambda = \frac{2}{3} \);
		\item \( \displaystyle A = \pares{-10, -5, 0}, ~ B = \pares{-2, -1, 4}, ~ \lambda = \frac{1}{4} \);
		\item \( \displaystyle A = \pares{-1, -3, -3}, ~ B = \pares{6, 4, -9}, ~ \lambda = \frac{1}{5} \);
		\item \( \displaystyle A = \pares{3, -1, 0}, ~ B = \pares{8, 3, -3}, ~ \lambda = \frac{4}{5} \);
		\item \( \displaystyle A = \pares{6, 9, -6}, ~ B = \pares{6, 0, 10}, ~ \lambda = \frac{1}{6} \);
		\item \( \displaystyle A = \pares{10, 8, 10}, ~ B = \pares{-10, -4, 5}, ~ \lambda = \frac{1}{7} \);
		\item \( \displaystyle A = \pares{-6, 7, 3}, ~ B = \pares{6, -9, -7}, ~ \lambda = \frac{4}{7} \);
		\item \( \displaystyle A = \pares{-4, 3, -5}, ~ B = \pares{0, -9, -9}, ~ \lambda = \frac{5}{7} \);
		\item \( \displaystyle A = \pares{2, 5, -3}, ~ B = \pares{5, 7, 8}, ~ \lambda = \frac{5}{8} \);

		\setcounter{tasks}{\value{enumi}}
	\end{enumerate}	
