\subsection{Задачи}

	Построить каноническое и параметрическое уравнения плоскости, проходящей через следующие точки:

	\begin{enumerate}
		\setcounter{enumi}{\value{tasks}}

			\item \( Q = \pares{2, 1, -3}, ~ S = \pares{3, 1, 4}, ~ J = \pares{4, -2, 0} \);
			\item \( U = \pares{-3, -3, -2}, ~ H = \pares{3, 4, -4}, ~ Z = \pares{-2, 2, 2} \);
			\item \( U = \pares{2, -1, -2}, ~ E = \pares{-3, 4, 4}, ~ L = \pares{-2, 3, -3} \);
			\item \( M = \pares{4, -4, 2}, ~ F = \pares{2, 1, 2}, ~ I = \pares{1, 2, 1} \);
			\item \( R = \pares{0, -1, 1}, ~ G = \pares{4, 4, 3}, ~ X = \pares{-3, -4, -4} \);
			\item \( W = \pares{-4, -2, 3}, ~ C = \pares{1, -4, 4}, ~ R = \pares{2, -2, -3} \);
			\item \( C = \pares{3, 3, -1}, ~ L = \pares{-2, 1, -4}, ~ O = \pares{-4, -1, 3} \);
			\item \( C = \pares{-1, 2, 2}, ~ P = \pares{-4, -2, 3}, ~ Q = \pares{0, -4, 3} \);
			\item \( M = \pares{0, 4, 1}, ~ L = \pares{4, -2, 0}, ~ P = \pares{-2, -1, 2} \);
			\item \( P = \pares{0, -1, 3}, ~ J = \pares{-2, -4, -4}, ~ R = \pares{0, -3, -3} \);
			\item \( J = \pares{2, -4, 2}, ~ K = \pares{2, -4, 4}, ~ P = \pares{2, -4, 4} \);
			\item \( B = \pares{1, 3, 3}, ~ K = \pares{-4, 0, -2}, ~ Y = \pares{-2, -4, -3} \);
			\item \( S = \pares{-4, -1, 2}, ~ B = \pares{-2, -3, 3}, ~ L = \pares{1, 1, 1} \);
			\item \( R = \pares{2, 2, 4}, ~ P = \pares{-4, -3, 2}, ~ S = \pares{3, -2, -3} \);
			\item \( J = \pares{2, 4, 1}, ~ T = \pares{3, 0, -1}, ~ S = \pares{3, 0, -2} \);
			\item \( O = \pares{-3, 4, -4}, ~ D = \pares{-1, 1, 2}, ~ W = \pares{0, 4, 4} \);
			\item \( R = \pares{-2, 2, -2}, ~ C = \pares{-1, -3, -4}, ~ I = \pares{-3, 0, 1} \);
			\item \( Q = \pares{2, 4, 0}, ~ K = \pares{-3, -2, 4}, ~ F = \pares{3, -3, 3} \);
			\item \( H = \pares{-4, -1, 0}, ~ C = \pares{2, -4, 1}, ~ O = \pares{0, 1, -1} \);
			\item \( T = \pares{2, -2, 1}, ~ A = \pares{2, -3, 2}, ~ P = \pares{1, -4, -4} \);

		\setcounter{tasks}{\value{enumi}}
	\end{enumerate}

	\vspace{15pt}
	Построить каноническое и параметрическое уравнения плоскости, основанные на следующих векторах:
	
	\begin{enumerate}
		\setcounter{enumi}{\value{tasks}}

			\item \( \displaystyle \va = \bracs{-9, -10, -9}, ~ \vb = \bracs{-3, -4, -9} \);
			\item \( \displaystyle \va = \bracs{8, 9, -4}, ~ \vb = \bracs{-3, 0, 8} \);
			\item \( \displaystyle \va = \bracs{-1, 10, -10}, ~ \vb = \bracs{8, 7, 3} \);
			\item \( \displaystyle \va = \bracs{8, -10, 0}, ~ \vb = \bracs{4, 1, 3} \);
			\item \( \displaystyle \va = \bracs{-5, -3, -9}, ~ \vb = \bracs{-3, 2, -4} \);
			\item \( \displaystyle \va = \bracs{5, 10, 3}, ~ \vb = \bracs{-3, -5, -7} \);
			\item \( \displaystyle \va = \bracs{-8, -1, -10}, ~ \vb = \bracs{0, -2, 6} \);
			\item \( \displaystyle \va = \bracs{3, 6, -9}, ~ \vb = \bracs{0, 1, 9} \);
			\item \( \displaystyle \va = \bracs{-9, -3, 2}, ~ \vb = \bracs{-4, -10, 7} \);
			\item \( \displaystyle \va = \bracs{-9, 9, -7}, ~ \vb = \bracs{-1, -2, -9} \);
			\item \( \displaystyle \va = \bracs{2, 0, 5}, ~ \vb = \bracs{1, -1, -6} \);
			\item \( \displaystyle \va = \bracs{5, -4, 4}, ~ \vb = \bracs{10, 8, -5} \);
			\item \( \displaystyle \va = \bracs{-9, -6, 2}, ~ \vb = \bracs{-1, 4, 7} \);
			\item \( \displaystyle \va = \bracs{-3, 7, 5}, ~ \vb = \bracs{5, 1, -4} \);
			\item \( \displaystyle \va = \bracs{5, 0, 3}, ~ \vb = \bracs{9, 5, -9} \);
			\item \( \displaystyle \va = \bracs{-4, -7, 3}, ~ \vb = \bracs{-4, -2, -10} \);
			\item \( \displaystyle \va = \bracs{-2, 6, 8}, ~ \vb = \bracs{7, 10, 5} \);
			\item \( \displaystyle \va = \bracs{-6, -1, 0}, ~ \vb = \bracs{2, 9, 1} \);
			\item \( \displaystyle \va = \bracs{1, 2, 8}, ~ \vb = \bracs{4, -2, 10} \);
			\item \( \displaystyle \va = \bracs{10, 9, -1}, ~ \vb = \bracs{-9, -5, 2} \);

		\setcounter{tasks}{\value{enumi}}
	\end{enumerate}

	\vspace{15pt}
	Построить уравнение плоскости, проходящее через точки, полученные проецированием точки $P$ на оси системы координат:

	\begin{multicols}{2}
		\begin{enumerate}
			\setcounter{enumi}{\value{tasks}}

				\item \( P = \pares{-8, -6, 2} \);
				\item \( P = \pares{8, 7, -5} \);
				\item \( P = \pares{4, -3, 4} \);
				\item \( P = \pares{-4, -7, -7} \);
				\item \( P = \pares{-9, -8, -2} \);
				\item \( P = \pares{-7, -7, 2} \);
				\item \( P = \pares{-10, -7, 4} \);
				\item \( P = \pares{2, -2, 8} \);
				\item \( P = \pares{6, 10, 10} \);
				\item \( P = \pares{-9, -9, -3} \);
				\item \( P = \pares{3, -7, 3} \);
				\item \( P = \pares{8, 5, 9} \);
				\item \( P = \pares{-10, 6, 8} \);
				\item \( P = \pares{9, -8, 10} \);
				\item \( P = \pares{-2, 9, -7} \);
				\item \( P = \pares{1, 5, 2} \);
				\item \( P = \pares{-6, -6, -10} \);
				\item \( P = \pares{-10, 5, -3} \);
				\item \( P = \pares{3, 7, 1} \);
				\item \( P = \pares{3, 10, 1} \);

			\setcounter{tasks}{\value{enumi}}
		\end{enumerate}
	\end{multicols}

	\pagebreak
	Построить уравнение плоскости, проходящее через точки $A$ и $B$, и перпендикулярную плоскости $p$:

	\begin{enumerate}
		\setcounter{enumi}{\value{tasks}}

			\item \( A = \pares{-4, -10, 2}, ~ B = \pares{8, -9, 5}, \quad p: 4 + y = 0 \);
			\item \( A = \pares{5, 2, -7}, ~ B = \pares{-3, -8, 4}, \quad p: 5 + y + 2z = 0 \);
			\item \( A = \pares{4, -5, 7}, ~ B = \pares{4, -6, 1}, \quad p: 6x - 19z + 11y = 0 \);
			\item \( A = \pares{-7, -10, -10}, ~ B = \pares{8, 4, 3}, \quad p: 3z + y + 4 + x = 0 \);
			\item \( A = \pares{-2, -7, 2}, ~ B = \pares{3, 2, 5}, \quad p: 2z + 1 + 2x - 3y = 0 \);
			\item \( A = \pares{8, 10, -2}, ~ B = \pares{1, -9, 0}, \quad p: 2x + 8y + 3z = 6 \);
			\item \( A = \pares{9, -5, 8}, ~ B = \pares{-5, 2, -6}, \quad p: 2y - 3x - z + 4 = 0 \);
			\item \( A = \pares{3, 2, 1}, ~ B = \pares{7, -8, -3}, \quad p: 6x + 5y + 4z + 22 = 0 \);
			\item \( A = \pares{3, -8, -6}, ~ B = \pares{4, 3, 10}, \quad p: 10x - 7z + 27 + 15y = 0 \);
			\item \( A = \pares{-3, -7, -9}, ~ B = \pares{6, -7, -9}, \quad p: 6z - x - 17y + 26 = 0 \);
			\item \( A = \pares{10, -7, 10}, ~ B = \pares{-8, -6, -1}, \quad p: 7y - z + 28 - 11x = 0 \);
			\item \( A = \pares{-6, -5, 4}, ~ B = \pares{-9, -3, -8}, \quad p: 12y + 38 - 9x - 28z = 0 \);
			\item \( A = \pares{-3, 4, -6}, ~ B = \pares{1, -8, 10}, \quad p: 16y + 30z + 17x + 26 = 0 \);
			\item \( A = \pares{0, 8, -10}, ~ B = \pares{5, 6, -10}, \quad p: 20 + 30z + x - 18y = 0 \);
			\item \( A = \pares{-6, 6, 4}, ~ B = \pares{-7, 0, 0}, \quad p: 16x - 12z + 31 - 17y = 0 \);
			\item \( A = \pares{7, 6, -4}, ~ B = \pares{0, 2, 5}, \quad p: 6y + 30z + 82 - 7x = 0 \);
			\item \( A = \pares{-4, 2, 0}, ~ B = \pares{9, 8, -10}, \quad p: 35z + 12y + 9 - 30x = 0 \);
			\item \( A = \pares{10, -4, -6}, ~ B = \pares{-4, 8, -10}, \quad p: 68 - x + 19z - 2y = 0 \);
			\item \( A = \pares{-1, 6, 2}, ~ B = \pares{0, 9, -9}, \quad p: 28z + 23x + y = 16 \);
			\item \( A = \pares{10, -10, -8}, ~ B = \pares{6, -5, 10}, \quad p: 11y - 9x + 4z = 24 \);

		\setcounter{tasks}{\value{enumi}}
	\end{enumerate}

	\pagebreak
	Вычислить угол между плоскостями, или показать, что они параллельны:

	\begin{enumerate}
		\setcounter{enumi}{\value{tasks}}

			\item \( y = 1, \quad 13y+4z-24x = 75 \);
			\item \( 6x-3y+10-5z = 0, \quad 1-y-x = 0 \);
			\item \( x+2-y = 0, \quad 13+y+4z-4x = 0 \);
			\item \( z+x-2y = 8, \quad 31z-17y+5x = 22 \);
			\item \( 3x+12+2y-2z = 0, \quad 8-3z-y = 0 \);
			\item \( 4x+16-3z-y = 0, \quad 4z-3x+7 = 0 \);
			\item \( 5x+34z+38y = 16, \quad 4z+3x-4y = 0 \);
			\item \( 5x-2z-6y = 10, \quad 16-16x-7z-10y = 0 \);
			\item \( 11x+38+23z+18y = 0, \quad 8+x+3y = 0 \);
			\item \( 6z-y+6x+14 = 0, \quad 20+12y+z+4x = 0 \);
			\item \( 12y+11z+32x+47 = 0, \quad 7z+2x+7y = 1 \);
			\item \( 32-35x+14z+15y = 0, \quad 8x+35z+48y = 20 \);

		\setcounter{tasks}{\value{enumi}}
	\end{enumerate}

	\vspace{10pt}
	\begin{enumerate}
		\setcounter{enumi}{\value{tasks}}

			\item \( \left\lbrace \begin{aligned}
						x &= -2v-1+4t \\
						y &= -5t-3v+4 \\
						z &= -2t+v-2
					\end{aligned} \right., \quad \left\lbrace \begin{aligned}
						x &= -4+7t+4v \\
						y &= -3+3v-t \\
						z &= -7t-3v+3
					\end{aligned} \right. \);
			\item \( \left\lbrace \begin{aligned}
						x &= v \\
						y &= t-v-1 \\
						z &= -4+7t+6v
					\end{aligned} \right., \quad \left\lbrace \begin{aligned}
						x &= -6t+4 \\
						y &= -4+7t+v \\
						z &= -3v+4-5t
					\end{aligned} \right. \);
			\item \( \left\lbrace \begin{aligned}
						x &= -2v+4-5t \\
						y &= -6v+2-5t \\
						z &= -5t+3-6v
					\end{aligned} \right., \quad \left\lbrace \begin{aligned}
						x &= -t-v \\
						y &= -2v+3 \\
						z &= t-4+4v
					\end{aligned} \right. \);
			\item \( \left\lbrace \begin{aligned}
						x &= -1-t \\
						y &= -5v+1 \\
						z &= -4
					\end{aligned} \right., \quad \left\lbrace \begin{aligned}
						x &= 6v-4 \\
						y &= 3v+6t-3 \\
						z &= v+2t-2
					\end{aligned} \right. \);
			\item \( \left\lbrace \begin{aligned}
						x &= 4v-3t \\
						y &= 4t \\
						z &= -t+3-5v
					\end{aligned} \right., \quad \left\lbrace \begin{aligned}
						x &= -3+t+v \\
						y &= 2+2t+v \\
						z &= 6t-3+4v
					\end{aligned} \right. \);
			\item \( \left\lbrace \begin{aligned}
						x &= 4-8t-2v \\
						y &= 6v+7t-3 \\
						z &= -2+4t+5v
					\end{aligned} \right., \quad \left\lbrace \begin{aligned}
						x &= -1+5t+3v \\
						y &= t \\
						z &= 4v-3+6t
					\end{aligned} \right. \);
			\item \( \left\lbrace \begin{aligned}
						x &= -1+2t+4v \\
						y &= 5t-1 \\
						z &= 2t-4v
					\end{aligned} \right., \quad \left\lbrace \begin{aligned}
						x &= -4v-3t+2 \\
						y &= 4-3t \\
						z &= 5v+7t-3
					\end{aligned} \right. \);
			\item \( \left\lbrace \begin{aligned}
						x &= 2-2v+2t \\
						y &= -3+7v \\
						z &= -4v-2t
					\end{aligned} \right., \quad \left\lbrace \begin{aligned}
						x &= 5t-3+7v \\
						y &= 3-v-5t \\
						z &= 4-8v-5t
					\end{aligned} \right. \);

		\setcounter{tasks}{\value{enumi}}
	\end{enumerate}

	\vspace{15pt}
	Найти точку пересечения прямой $l$ и плоскости $p$, или, если такой не существует -- расстояние между ними:

	\begin{enumerate}
		\setcounter{enumi}{\value{tasks}}

			\item \( \displaystyle l: \frac{x-3}{0} = \frac{-9+y}{-5} = \frac{1+z}{6}, \quad p: z = 0 \);
			\item \( \displaystyle l: \frac{3+x}{11} = \frac{y+9}{14} = \frac{7+z}{14}, \quad p: y = 3 \);
			\item \( \displaystyle l: \frac{x+6}{12} = \frac{y}{-1} = \frac{3+z}{0}, \quad p: 40+3z-4x+9y = 0 \);
			\item \( \displaystyle l: \frac{8+x}{3} = \frac{2+y}{8} = \frac{-4+z}{-3}, \quad p: 26-5x+18z-11y = 0 \);
			\item \( \displaystyle l: \frac{-6+x}{-8} = \frac{9+y}{15} = \frac{z+5}{12}, \quad p: 8x+2+5y+2z = 0 \);
			\item \( \displaystyle l: \frac{4+x}{6} = \frac{7+y}{-3} = \frac{-1+z}{3}, \quad p: 2x-5y+11z = 21 \);
			\item \( \displaystyle l: \frac{x+4}{-2} = \frac{3+y}{9} = \frac{-8+z}{-2}, \quad p: 19x-5z-15y+2 = 0 \);
			\item \( \displaystyle l: \frac{x+1}{2} = \frac{y-5}{4} = \frac{z-2}{8}, \quad p: 23z+2y+14x = 2 \);
			\item \( \displaystyle l: \frac{x-7}{3} = \frac{y+3}{-4} = \frac{-5+z}{-4}, \quad p: 6z-y+4 = 0 \);
			\item \( \displaystyle l: \frac{8+x}{5} = \frac{4+y}{-6} = \frac{z-1}{-5}, \quad p: 6-y-2x = 0 \);
			\item \( \displaystyle l: \frac{x+3}{-6} = \frac{-1+y}{-4} = \frac{-5+z}{-13}, \quad p: 4x-5z+20+2y = 0 \);
			\item \( \displaystyle l: \frac{x+1}{6} = \frac{y-3}{-8} = \frac{2+z}{7}, \quad p: 7x-z+10y = 40 \);

		\setcounter{tasks}{\value{enumi}}
	\end{enumerate}

	\vspace{10pt}
	\begin{enumerate}
		\setcounter{enumi}{\value{tasks}}

			\item \( l: \left\lbrace \begin{aligned}
						x &= 8+2t \\
						y &= 2t+2 \\
						z &= t+1
					\end{aligned} \right., \quad p: \left\lbrace \begin{aligned}
						x &= 7v-4+8t \\
						y &= 3-4t-3v \\
						z &= 4-t-3v
					\end{aligned} \right. \);
			\item \( l: \left\lbrace \begin{aligned}
						x &= -6 \\
						y &= t \\
						z &= -3
					\end{aligned} \right., \quad p: \left\lbrace \begin{aligned}
						x &= 3-6t-5v \\
						y &= 2-3v+2t \\
						z &= 6v-3
					\end{aligned} \right. \);
			\item \( l: \left\lbrace \begin{aligned}
						x &= -7t+9 \\
						y &= 6t-5 \\
						z &= -11t+8
					\end{aligned} \right., \quad p: \left\lbrace \begin{aligned}
						x &= -2+3v+2t \\
						y &= 2v+1 \\
						z &= 2-3t
					\end{aligned} \right. \);
			\item \( l: \left\lbrace \begin{aligned}
						x &= 8t-5 \\
						y &= -9t+4 \\
						z &= -1+6t
					\end{aligned} \right., \quad p: \left\lbrace \begin{aligned}
						x &= -2+t \\
						y &= -1-2v+2t \\
						z &= 2+v
					\end{aligned} \right. \);
			\item \( l: \left\lbrace \begin{aligned}
						x &= -3t+4 \\
						y &= -14t+7 \\
						z &= t+5
					\end{aligned} \right., \quad p: \left\lbrace \begin{aligned}
						x &= 4t-2+3v \\
						y &= -3t+1+2v \\
						z &= v+1+3t
					\end{aligned} \right. \);
			\item \( l: \left\lbrace \begin{aligned}
						x &= -5+11t \\
						y &= -2t-1 \\
						z &= -10+3t
					\end{aligned} \right., \quad p: \left\lbrace \begin{aligned}
						x &= -4+4t+2v \\
						y &= -v-2t \\
						z &= t+4v
					\end{aligned} \right. \);
			\item \( l: \left\lbrace \begin{aligned}
						x &= -3-7t \\
						y &= 10-14t \\
						z &= -1+9t
					\end{aligned} \right., \quad p: \left\lbrace \begin{aligned}
						x &= -2+6v+2t \\
						y &= 2v-3 \\
						z &= -3v+2-5t
					\end{aligned} \right. \);
			\item \( l: \left\lbrace \begin{aligned}
						x &= -10+7t \\
						y &= -2-8t \\
						z &= -1-2t
					\end{aligned} \right., \quad p: \left\lbrace \begin{aligned}
						x &= 3-3t-2v \\
						y &= -4v-4t+2 \\
						z &= 4v-2t
					\end{aligned} \right. \);

		\setcounter{tasks}{\value{enumi}}
	\end{enumerate}

	\vspace{15pt}
	Найти расстояние между прямыми в трехмерном пространстве:

	\begin{enumerate}
		\setcounter{enumi}{\value{tasks}}
		
			\item \( \displaystyle \frac{x}{2} = \frac{2+y}{-6} = \frac{-8+z}{-3}, \quad \frac{2+x}{0} = \frac{-1+y}{-6} = \frac{z+1}{10} \);
			\item \( \displaystyle \frac{x+6}{8} = \frac{y+9}{6} = \frac{z-9}{-10}, \quad \frac{x+8}{11} = \frac{-10+y}{-13} = \frac{-1+z}{-2} \);
			\item \( \displaystyle \frac{8+x}{18} = \frac{y-6}{-5} = \frac{z-9}{-15}, \quad \frac{x-7}{-2} = \frac{y-9}{-11} = \frac{z-4}{-3} \);
			\item \( \displaystyle \frac{x}{10} = \frac{y-6}{2} = \frac{z-10}{-10}, \quad \frac{x-2}{6} = \frac{-7+y}{0} = \frac{z-8}{-8} \);
			\item \( \displaystyle \frac{x-5}{2} = \frac{y+4}{-6} = \frac{2+z}{-1}, \quad \frac{x-9}{1} = \frac{1+y}{4} = \frac{z+10}{20} \);
			\item \( \displaystyle \frac{4+x}{12} = \frac{10+y}{10} = \frac{z-10}{-18}, \quad \frac{x+5}{9} = \frac{y+7}{16} = \frac{-10+z}{-8} \);
			\item \( \displaystyle \frac{x}{4} = \frac{8+y}{13} = \frac{z+10}{4}, \quad \frac{x-4}{5} = \frac{y+3}{8} = \frac{z-3}{-4} \);
			\item \( \displaystyle \frac{x+8}{6} = \frac{1+y}{10} = \frac{z+7}{0}, \quad \frac{x+10}{19} = \frac{3+y}{-6} = \frac{-5+z}{4} \);
			\item \( \displaystyle \frac{x+7}{10} = \frac{-10+y}{-12} = \frac{-7+z}{-3}, \quad \frac{2+x}{-2} = \frac{-3+y}{-8} = \frac{3+z}{6} \);
			\item \( \displaystyle \frac{x+7}{0} = \frac{y+10}{16} = \frac{-9+z}{-1}, \quad \frac{9+x}{16} = \frac{-2+y}{-3} = \frac{z+6}{8} \);
		
		\setcounter{tasks}{\value{enumi}}
	\end{enumerate}