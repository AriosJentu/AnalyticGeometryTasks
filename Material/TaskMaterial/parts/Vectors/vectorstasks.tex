\subsection{Задачи}

	Для следующих пар двумерных векторов $\va, \vb = \bracs{x, y}$, найти их сумму и разность, и построить в системе координат $Oxy$:

	\begin{multicols}{2}
		\begin{enumerate}
			\setcounter{enumi}{\value{tasks}}

			\item \( \displaystyle \va = \bracs{7, -7}, ~ \vb = \bracs{-7, 5} \);
			\item \( \displaystyle \va = \bracs{-8, -3}, ~ \vb = \bracs{-2, 9} \);
			\item \( \displaystyle \va = \bracs{-1, -5}, ~ \vb = \bracs{-6, 0} \);
			\item \( \displaystyle \va = \bracs{0, 6}, ~ \vb = \bracs{2, -6} \);
			\item \( \displaystyle \va = \bracs{3, 9}, ~ \vb = \bracs{-9, 2} \);
			\item \( \displaystyle \va = \bracs{-6, -9}, ~ \vb = \bracs{2, 6} \);
			\item \( \displaystyle \va = \bracs{5, -1}, ~ \vb = \bracs{-1, 5} \);
			\item \( \displaystyle \va = \bracs{6, 0}, ~ \vb = \bracs{-6, -10} \);
			\item \( \displaystyle \va = \bracs{2, -8}, ~ \vb = \bracs{-5, 3} \);
			\item \( \displaystyle \va = \bracs{3, 1}, ~ \vb = \bracs{-5, 8} \);
			\item \( \displaystyle \va = \bracs{-5, 9}, ~ \vb = \bracs{5, 7} \);
			\item \( \displaystyle \va = \bracs{-3, -8}, ~ \vb = \bracs{-4, 3} \);
			\item \( \displaystyle \va = \bracs{-6, -8}, ~ \vb = \bracs{8, -4} \);
			\item \( \displaystyle \va = \bracs{-6, -7}, ~ \vb = \bracs{-3, 8} \);
			\item \( \displaystyle \va = \bracs{-5, -1}, ~ \vb = \bracs{-2, 7} \);
			\item \( \displaystyle \va = \bracs{9, -1}, ~ \vb = \bracs{-4, 9} \);
			\item \( \displaystyle \va = \bracs{-1, -1}, ~ \vb = \bracs{0, 9} \);
			\item \( \displaystyle \va = \bracs{2, 10}, ~ \vb = \bracs{-6, -5} \);
			\item \( \displaystyle \va = \bracs{10, 1}, ~ \vb = \bracs{-3, -2} \);
			\item \( \displaystyle \va = \bracs{-5, -4}, ~ \vb = \bracs{2, -4} \);
			\item \( \displaystyle \va = \bracs{4, 6}, ~ \vb = \bracs{10, 9} \);
			\item \( \displaystyle \va = \bracs{1, 2}, ~ \vb = \bracs{-8, 9} \);

			\setcounter{tasks}{\value{enumi}}
		\end{enumerate}
	\end{multicols}

	\vspace{15pt}
	Для следующих пар трехмерных векторов $\va, \vb = \bracs{x, y, z}$, найти их сумму и разность:

	\begin{enumerate}
		\setcounter{enumi}{\value{tasks}}

		\item \( \displaystyle \va = \bracs{-8, 3, 3}, ~ \vb = \bracs{10, 8, -5} \);
		\item \( \displaystyle \va = \bracs{5, -8, -4}, ~ \vb = \bracs{-5, -10, -6} \);
		\item \( \displaystyle \va = \bracs{-2, -5, 2}, ~ \vb = \bracs{6, -4, 6} \);
		\item \( \displaystyle \va = \bracs{10, 6, 10}, ~ \vb = \bracs{6, 0, 0} \);
		\item \( \displaystyle \va = \bracs{0, 1, 1}, ~ \vb = \bracs{4, -10, 9} \);
		\item \( \displaystyle \va = \bracs{-8, -7, -3}, ~ \vb = \bracs{-10, 8, -4} \);
		\item \( \displaystyle \va = \bracs{-1, 0, 2}, ~ \vb = \bracs{4, -5, -2} \);
		\item \( \displaystyle \va = \bracs{-1, 5, -9}, ~ \vb = \bracs{-4, -10, 3} \);
		\item \( \displaystyle \va = \bracs{6, -4, 2}, ~ \vb = \bracs{6, -2, -10} \);
		\item \( \displaystyle \va = \bracs{-6, -3, -2}, ~ \vb = \bracs{1, 1, 5} \);
		\item \( \displaystyle \va = \bracs{1, -9, 1}, ~ \vb = \bracs{5, 8, -2} \);
		\item \( \displaystyle \va = \bracs{-5, -1, 6}, ~ \vb = \bracs{3, 4, 3} \);
		\item \( \displaystyle \va = \bracs{-6, -7, 1}, ~ \vb = \bracs{2, 3, -3} \);
		\item \( \displaystyle \va = \bracs{10, 3, 0}, ~ \vb = \bracs{-4, -2, -6} \);
		\item \( \displaystyle \va = \bracs{-8, 2, 5}, ~ \vb = \bracs{-5, 7, 7} \);
		\item \( \displaystyle \va = \bracs{-5, 7, 10}, ~ \vb = \bracs{6, 6, -9} \);

		\setcounter{tasks}{\value{enumi}}
	\end{enumerate}

	\vspace{15pt}
	Найти значение вектора $\vc$, если известны следующие выражения:

	\begin{enumerate}
		\setcounter{enumi}{\value{tasks}}

			\item \( \displaystyle \vc = - 2 \cdot \va - 4 \cdot \vb, ~ \va = \bracs{2, 5, -4}, ~ \vb = \bracs{2, 5, 9} \);
			\item \( \displaystyle \vc = - 4 \cdot \va - 1 \cdot \vb, ~ \va = \bracs{10, -2, -3}, ~ \vb = \bracs{8, -6, -5} \);
			\item \( \displaystyle \vc = 7 \cdot \va + 3 \cdot \vb, ~ \va = \bracs{-5, -3, 4}, ~ \vb = \bracs{0, -7, 4} \);
			\item \( \displaystyle \vc = 7 \cdot \va - 3 \cdot \vb, ~ \va = \bracs{-3, -5, -5}, ~ \vb = \bracs{2, -3, -9} \);
			\item \( \displaystyle \vc = 8 \cdot \va - 4 \cdot \vb, ~ \va = \bracs{-7, 6, -9}, ~ \vb = \bracs{-1, 7, 2} \);
			\item \( \displaystyle \vc = 4 \cdot \va - 7 \cdot \vb, ~ \va = \bracs{-3, -7, 5}, ~ \vb = \bracs{-7, 9, 10} \);
			\item \( \displaystyle \vc = - 6 \cdot \va + 4 \cdot \vb, ~ \va = \bracs{-5, 1, -2}, ~ \vb = \bracs{0, -10, -7} \);
			\item \( \displaystyle \vc = 5 \cdot \va - 6 \cdot \vb, ~ \va = \bracs{10, -4, -3}, ~ \vb = \bracs{-1, 2, 5} \);
			\item \( \displaystyle \vc = - 4 \cdot \va - 2 \cdot \vb, ~ \va = \bracs{4, 4, -4}, ~ \vb = \bracs{1, 6, 7} \);
			\item \( \displaystyle \vc = 8 \cdot \va + \vb, ~ \va = \bracs{0, -10, -9}, ~ \vb = \bracs{7, -4, 1} \);
			\item \( \displaystyle \vc = - 5 \cdot \va + 6 \cdot \vb, ~ \va = \bracs{-9, -1, -2}, ~ \vb = \bracs{8, 10, -6} \);
			\item \( \displaystyle \vc = - \va - 7 \cdot \vb, ~ \va = \bracs{-2, -3, 1}, ~ \vb = \bracs{10, 2, 6} \);
			\item \( \displaystyle \vc = \va + 3 \cdot \vb, ~ \va = \bracs{7, 6, 8}, ~ \vb = \bracs{9, 9, 9} \);
			\item \( \displaystyle \vc = 7 \cdot \va - 3 \cdot \vb, ~ \va = \bracs{-5, 2, -8}, ~ \vb = \bracs{6, -1, 3} \);
			\item \( \displaystyle \vc = \va + 4 \cdot \vb, ~ \va = \bracs{-8, 5, 2}, ~ \vb = \bracs{5, 1, 1} \);
			\item \( \displaystyle \vc = 9 \cdot \va + 4 \cdot \vb, ~ \va = \bracs{6, -4, 10}, ~ \vb = \bracs{8, -9, -9} \);
			\item \( \displaystyle \vc = 9 \cdot \va + 7 \cdot \vb, ~ \va = \bracs{5, -10, -10}, ~ \vb = \bracs{-1, 5, 2} \);
			\item \( \displaystyle \vc = - 2 \cdot \va + 8 \cdot \vb, ~ \va = \bracs{2, 6, 10}, ~ \vb = \bracs{2, -10, -1} \);
			\item \( \displaystyle \vc = - \va + 8 \cdot \vb, ~ \va = \bracs{8, -3, -3}, ~ \vb = \bracs{8, -8, 5} \);
			\item \( \displaystyle \vc = - 4 \cdot \va - 3 \cdot \vb, ~ \va = \bracs{1, 2, 4}, ~ \vb = \bracs{0, -3, -2} \);

		\setcounter{tasks}{\value{enumi}}
	\end{enumerate}

	\pagebreak
	Построить нормированный базис $\ve_1, \ve_2$, в основании которого лежит следующая система векторов:

	\begin{multicols}{2}
		\begin{enumerate}
			\setcounter{enumi}{\value{tasks}}

			\item \( \displaystyle \vf_1 = \bracs{2, -9}, ~ \vf_2 = \bracs{-7, 7} \);
			\item \( \displaystyle \vf_1 = \bracs{8, 10}, ~ \vf_2 = \bracs{9, -4} \);
			\item \( \displaystyle \vf_1 = \bracs{5, 1}, ~ \vf_2 = \bracs{9, 4} \);
			\item \( \displaystyle \vf_1 = \bracs{8, -6}, ~ \vf_2 = \bracs{-6, 9} \);
			\item \( \displaystyle \vf_1 = \bracs{2, -8}, ~ \vf_2 = \bracs{-4, 4} \);
			\item \( \displaystyle \vf_1 = \bracs{-2, 5}, ~ \vf_2 = \bracs{7, -5} \);
			\item \( \displaystyle \vf_1 = \bracs{-3, 5}, ~ \vf_2 = \bracs{1, 3} \);
			\item \( \displaystyle \vf_1 = \bracs{-4, 3}, ~ \vf_2 = \bracs{-8, 8} \);
			\item \( \displaystyle \vf_1 = \bracs{-8, -6}, ~ \vf_2 = \bracs{1, 1} \);
			\item \( \displaystyle \vf_1 = \bracs{-4, 10}, ~ \vf_2 = \bracs{5, 3} \);
			\item \( \displaystyle \vf_1 = \bracs{9, -5}, ~ \vf_2 = \bracs{-2, 6} \);
			\item \( \displaystyle \vf_1 = \bracs{5, 9}, ~ \vf_2 = \bracs{-9, 10} \);
			\item \( \displaystyle \vf_1 = \bracs{-5, 4}, ~ \vf_2 = \bracs{6, -7} \);
			\item \( \displaystyle \vf_1 = \bracs{5, -2}, ~ \vf_2 = \bracs{-2, 4} \);
			\item \( \displaystyle \vf_1 = \bracs{-6, -5}, ~ \vf_2 = \bracs{3, 3} \);
			\item \( \displaystyle \vf_1 = \bracs{6, 1}, ~ \vf_2 = \bracs{9, 8} \);
			\item \( \displaystyle \vf_1 = \bracs{1, 4}, ~ \vf_2 = \bracs{7, 9} \);
			\item \( \displaystyle \vf_1 = \bracs{5, 8}, ~ \vf_2 = \bracs{8, -1} \);

			\setcounter{tasks}{\value{enumi}}
		\end{enumerate}
	\end{multicols}

	\vspace{15pt}
	Найти координаты вектора $\va$ из декартовой системы координат в линейном базисе $\ve_1, \ve_2$:

	\begin{enumerate}
		\setcounter{enumi}{\value{tasks}}

		\item \( \displaystyle \va = \bracs{-9, 9}, \quad \ve_1 = \bracs{-7, 7}, ~ \ve_2 = \bracs{4, 4} \);
		\item \( \displaystyle \va = \bracs{2, -7}, \quad \ve_1 = \bracs{5, 10}, ~ \ve_2 = \bracs{-5, 7} \);
		\item \( \displaystyle \va = \bracs{6, -4}, \quad \ve_1 = \bracs{-8, 2}, ~ \ve_2 = \bracs{-3, -4} \);
		\item \( \displaystyle \va = \bracs{8, -8}, \quad \ve_1 = \bracs{-9, 8}, ~ \ve_2 = \bracs{7, 10} \);
		\item \( \displaystyle \va = \bracs{-8, -3}, \quad \ve_1 = \bracs{8, 3}, ~ \ve_2 = \bracs{-9, -3} \);
		\item \( \displaystyle \va = \bracs{-6, 0}, \quad \ve_1 = \bracs{-1, -2}, ~ \ve_2 = \bracs{7, -9} \);
		\item \( \displaystyle \va = \bracs{-5, -1}, \quad \ve_1 = \bracs{-8, 2}, ~ \ve_2 = \bracs{1, -3} \);
		\item \( \displaystyle \va = \bracs{-6, 4}, \quad \ve_1 = \bracs{8, 9}, ~ \ve_2 = \bracs{1, -3} \);
		\item \( \displaystyle \va = \bracs{-3, 0}, \quad \ve_1 = \bracs{5, 8}, ~ \ve_2 = \bracs{-3, 1} \);
		\item \( \displaystyle \va = \bracs{-5, 3}, \quad \ve_1 = \bracs{-2, -5}, ~ \ve_2 = \bracs{2, -2} \);
		\item \( \displaystyle \va = \bracs{5, 8}, \quad \ve_1 = \bracs{1, 9}, ~ \ve_2 = \bracs{-5, -2} \);
		\item \( \displaystyle \va = \bracs{0, -3}, \quad \ve_1 = \bracs{5, 0}, ~ \ve_2 = \bracs{2, 1} \);
		\item \( \displaystyle \va = \bracs{2, 3}, \quad \ve_1 = \bracs{5, -6}, ~ \ve_2 = \bracs{-4, 1} \);
		\item \( \displaystyle \va = \bracs{9, 9}, \quad \ve_1 = \bracs{5, 5}, ~ \ve_2 = \bracs{2, 9} \);
		\item \( \displaystyle \va = \bracs{-10, 2}, \quad \ve_1 = \bracs{2, 10}, ~ \ve_2 = \bracs{-8, 5} \);
		\item \( \displaystyle \va = \bracs{5, -1}, \quad \ve_1 = \bracs{0, -9}, ~ \ve_2 = \bracs{-6, -5} \);
		\item \( \displaystyle \va = \bracs{4, 0}, \quad \ve_1 = \bracs{-6, 8}, ~ \ve_2 = \bracs{-4, 0} \);
		\item \( \displaystyle \va = \bracs{-6, -3}, \quad \ve_1 = \bracs{-6, -5}, ~ \ve_2 = \bracs{-10, 5} \);

		\setcounter{tasks}{\value{enumi}}
	\end{enumerate}

	\vspace{15pt}
	Найти длины двух векторов и угол между ними в двумерном пространстве:

	\begin{multicols}{2}
		\begin{enumerate}
			\setcounter{enumi}{\value{tasks}}

				\item \( \displaystyle \va = \bracs{1, -3}, ~ \vb = \bracs{3, -1} \);
				\item \( \displaystyle \va = \bracs{-1, -5}, ~ \vb = \bracs{5, 1} \);
				\item \( \displaystyle \va = \bracs{0, -2}, ~ \vb = \bracs{-1, 4} \);
				\item \( \displaystyle \va = \bracs{-3, 1}, ~ \vb = \bracs{-2, 4} \);
				\item \( \displaystyle \va = \bracs{-2, 5}, ~ \vb = \bracs{2, -4} \);
				\item \( \displaystyle \va = \bracs{3, 1}, ~ \vb = \bracs{0, 4} \);
				\item \( \displaystyle \va = \bracs{3, -2}, ~ \vb = \bracs{-4, -4} \);
				\item \( \displaystyle \va = \bracs{0, -2}, ~ \vb = \bracs{-5, 2} \);
				\item \( \displaystyle \va = \bracs{-1, 1}, ~ \vb = \bracs{3, -3} \);
				\item \( \displaystyle \va = \bracs{-5, 0}, ~ \vb = \bracs{1, -5} \);
				\item \( \displaystyle \va = \bracs{1, 4}, ~ \vb = \bracs{4, -2} \);
				\item \( \displaystyle \va = \bracs{-1, 2}, ~ \vb = \bracs{1, -1} \);
				\item \( \displaystyle \va = \bracs{2, -2}, ~ \vb = \bracs{4, 0} \);
				\item \( \displaystyle \va = \bracs{-2, -3}, ~ \vb = \bracs{-4, 5} \);
				\item \( \displaystyle \va = \bracs{3, 3}, ~ \vb = \bracs{-5, 2} \);
				\item \( \displaystyle \va = \bracs{-1, -5}, ~ \vb = \bracs{5, -4} \);

			\setcounter{tasks}{\value{enumi}}
		\end{enumerate}
	\end{multicols}

	\vspace{15pt}
	Найти скалярное произведение двух векторов в двумерном пространстве, если известны их длины и косинус угла между ними:
	
	\begin{enumerate}
		\setcounter{enumi}{\value{tasks}}
	
			\item \( \displaystyle \va \cdot \vb, ~ \babs{\va} = 3, ~ \babs{\vb} = 5, \quad \cos{\alpha} = -\frac{3}{5} \);
			\item \( \displaystyle \va \cdot \vb, ~ \babs{\va} = 3, ~ \babs{\vb} = 5, \quad \cos{\alpha} = \frac{4}{5} \);
			\item \( \displaystyle \va \cdot \vb, ~ \babs{\va} = 2, ~ \babs{\vb} = \sqrt{2}, \quad \cos{\alpha} = -\frac{2}{ \sqrt{8} } \);
			\item \( \displaystyle \va \cdot \vb, ~ \babs{\va} = \sqrt{8}, ~ \babs{\vb} = \sqrt{5}, \quad \cos{\alpha} = \frac{2}{ \sqrt{40} } \);
			\item \( \displaystyle \va \cdot \vb, ~ \babs{\va} = \sqrt{10}, ~ \babs{\vb} = \sqrt{5}, \quad \cos{\alpha} = \frac{1}{ \sqrt{50} } \);
			\item \( \displaystyle \va \cdot \vb, ~ \babs{\va} = \sqrt{5}, ~ \babs{\vb} = 5, \quad \cos{\alpha} = \frac{10}{ \sqrt{125} } \);
			\item \( \displaystyle \va \cdot \vb, ~ \babs{\va} = \sqrt{5}, ~ \babs{\vb} = \sqrt{34}, \quad \cos{\alpha} = -\frac{7}{ \sqrt{170} } \);
			\item \( \displaystyle \va \cdot \vb, ~ \babs{\va} = \sqrt{10}, ~ \babs{\vb} = 5, \quad \cos{\alpha} = -\frac{9}{ \sqrt{250} } \);
			\item \( \displaystyle \va \cdot \vb, ~ \babs{\va} = \sqrt{20}, ~ \babs{\vb} = \sqrt{17}, \quad \cos{\alpha} = \frac{12}{ \sqrt{340} } \);
			\item \( \displaystyle \va \cdot \vb, ~ \babs{\va} = \sqrt{20}, ~ \babs{\vb} = \sqrt{17}, \quad \cos{\alpha} = -\frac{14}{ \sqrt{340} } \);
			\item \( \displaystyle \va \cdot \vb, ~ \babs{\va} = 5, ~ \babs{\vb} = \sqrt{17}, \quad \cos{\alpha} = \frac{19}{ \sqrt{425} } \);
			\item \( \displaystyle \va \cdot \vb, ~ \babs{\va} = \sqrt{13}, ~ \babs{\vb} = \sqrt{34}, \quad \cos{\alpha} = \frac{21}{ \sqrt{442} } \);
			\item \( \displaystyle \va \cdot \vb, ~ \babs{\va} = \sqrt{32}, ~ \babs{\vb} = 4, \quad \cos{\alpha} = -\frac{16}{ \sqrt{512} } \);
			\item \( \displaystyle \va \cdot \vb, ~ \babs{\va} = 5, ~ \babs{\vb} = \sqrt{34}, \quad \cos{\alpha} = -\frac{29}{ \sqrt{850} } \);

		\setcounter{tasks}{\value{enumi}}
	\end{enumerate}

	\vspace{15pt}
	Найти скалярное произведение двух векторов:

	\begin{enumerate}
		\setcounter{enumi}{\value{tasks}}

		\item \( \displaystyle \va = \bracs{7, -5, -1}, ~ \vb = \bracs{-10, -3, 2} \);
		\item \( \displaystyle \va = \bracs{6, 3, -6}, ~ \vb = \bracs{6, -7, 7} \);
		\item \( \displaystyle \va = \bracs{10, 9, -4}, ~ \vb = \bracs{1, -7, -5} \);
		\item \( \displaystyle \va = \bracs{1, 3, -6}, ~ \vb = \bracs{-10, 10, 4} \);
		\item \( \displaystyle \va = \bracs{4, 7, -10}, ~ \vb = \bracs{10, -10, -5} \);
		\item \( \displaystyle \va = \bracs{4, -4, 3}, ~ \vb = \bracs{-7, 4, -5} \);
		\item \( \displaystyle \va = \bracs{5, 2, 2}, ~ \vb = \bracs{-4, 9, -4} \);
		\item \( \displaystyle \va = \bracs{-1, 6, -9}, ~ \vb = \bracs{-5, 5, -5} \);
		\item \( \displaystyle \va = \bracs{-4, 6, -10}, ~ \vb = \bracs{5, -9, -7} \);
		\item \( \displaystyle \va = \bracs{5, 8, 4}, ~ \vb = \bracs{-5, 4, -8} \);
		\item \( \displaystyle \va = \bracs{5, 3, 7}, ~ \vb = \bracs{-8, 8, 0} \);
		\item \( \displaystyle \va = \bracs{-10, -7, 9}, ~ \vb = \bracs{-4, -8, -2} \);
		\item \( \displaystyle \va = \bracs{9, 10, 6}, ~ \vb = \bracs{-3, 10, -7} \);
		\item \( \displaystyle \va = \bracs{10, -9, 6}, ~ \vb = \bracs{-8, 8, -7} \);
		\item \( \displaystyle \va = \bracs{-1, -3, 4}, ~ \vb = \bracs{-2, 4, 10} \);
		\item \( \displaystyle \va = \bracs{0, 3, 2}, ~ \vb = \bracs{10, -10, -4} \);

		\setcounter{tasks}{\value{enumi}}
	\end{enumerate}

	\vspace{15pt}
	Найти косинус угла между векторами, если известны их длины и скалярное произведение:

	\begin{enumerate}
		\setcounter{enumi}{\value{tasks}}

			\item \( \displaystyle \babs{\va} = 4, ~ \babs{\vb} = 2, ~ \va \cdot \vb = -8 \);
			\item \( \displaystyle \babs{\va} = 5, ~ \babs{\vb} = 3, ~ \va \cdot \vb = 0 \);
			\item \( \displaystyle \babs{\va} = 1, ~ \babs{\vb} = \sqrt{41}, ~ \va \cdot \vb = -4 \);
			\item \( \displaystyle \babs{\va} = 2, ~ \babs{\vb} = \sqrt{2}, ~ \va \cdot \vb = -2 \);
			\item \( \displaystyle \babs{\va} = 6, ~ \babs{\vb} = \sqrt{26}, ~ \va \cdot \vb = -14 \);
			\item \( \displaystyle \babs{\va} = \sqrt{45}, ~ \babs{\vb} = 1, ~ \va \cdot \vb = 5 \);
			\item \( \displaystyle \babs{\va} = \sqrt{10}, ~ \babs{\vb} = 5, ~ \va \cdot \vb = 13 \);
			\item \( \displaystyle \babs{\va} = \sqrt{20}, ~ \babs{\vb} = 5, ~ \va \cdot \vb = -20 \);
			\item \( \displaystyle \babs{\va} = \sqrt{29}, ~ \babs{\vb} = 5, ~ \va \cdot \vb = -26 \);
			\item \( \displaystyle \babs{\va} = \sqrt{35}, ~ \babs{\vb} = 6, ~ \va \cdot \vb = 6 \);
			\item \( \displaystyle \babs{\va} = \sqrt{5}, ~ \babs{\vb} = \sqrt{13}, ~ \va \cdot \vb = 3 \);
			\item \( \displaystyle \babs{\va} = \sqrt{13}, ~ \babs{\vb} = \sqrt{10}, ~ \va \cdot \vb = 9 \);
			\item \( \displaystyle \babs{\va} = \sqrt{14}, ~ \babs{\vb} = \sqrt{27}, ~ \va \cdot \vb = 12 \);
			\item \( \displaystyle \babs{\va} = \sqrt{26}, ~ \babs{\vb} = \sqrt{17}, ~ \va \cdot \vb = 19 \);
			\item \( \displaystyle \babs{\va} = \sqrt{34}, ~ \babs{\vb} = \sqrt{13}, ~ \va \cdot \vb = -1 \);
			\item \( \displaystyle \babs{\va} = \sqrt{35}, ~ \babs{\vb} = \sqrt{27}, ~ \va \cdot \vb = 21 \);
			\item \( \displaystyle \babs{\va} = \sqrt{33}, ~ \babs{\vb} = \sqrt{30}, ~ \va \cdot \vb = -26 \);
			\item \( \displaystyle \babs{\va} = \sqrt{27}, ~ \babs{\vb} = \sqrt{38}, ~ \va \cdot \vb = -24 \);
			\item \( \displaystyle \babs{\va} = \sqrt{38}, ~ \babs{\vb} = \sqrt{35}, ~ \va \cdot \vb = 2 \);
			\item \( \displaystyle \babs{\va} = \sqrt{41}, ~ \babs{\vb} = \sqrt{20}, ~ \va \cdot \vb = 26 \);
			
		\setcounter{tasks}{\value{enumi}}
	\end{enumerate}
