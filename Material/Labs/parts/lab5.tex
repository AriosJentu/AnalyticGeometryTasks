В пятой лабораторной работе мы займемся отрисовкой графики в консоли, реализацией системы событий, и системой настроек.

\subsection{Система событий}
	
	Рассматривать будем примитивную систему событий, в которой будет только вызов и обработка событий в некотором пространстве событий. Сама система событий представляет собой систему обмена информацией между независимыми объектами, и задается в каждом объекте класса \inlinecode{Game} индивидуально. События можно вызывать с некоторыми параметрами, которые отлавливаются самой системой, и обрабатываются.

	Работа производится по следующему принципу:
	\begin{enumerate}
		\item Произвольная сущность вызывает событие с некоторыми аргументами;
		\item Обработчик событий вызывает функции, отвечающие за данное событие с предоставленными ему аргументами.
	\end{enumerate}
	% Для конкретных событий существует массив функций, которые вызываются последовательно внутри игры с аргументами, переданными обработчику. 
	В данном варианте системы событий, обращение к обработчику производится по имени события.

	Рассмотрим следующий пример. Положим, что существует объект \inlinecode{sphere}, который при перемещении камеры игрока \inlinecode{player} двигается вместе с ним. Для камеры существует обработчик нажатия на клавиши перемещения. Введем новое событие (\inlinecode{Game.Event.add}) в систему под названием \inlinecode{\"OnMainCameraMove\"}. Для данного события добавим функцию-обработчик (\inlinecode{Game.Event.handle}), которая принимает на себя в качестве аргументов сущность игрока, и меняет положение сферы. Затем вызовем событие (\inlinecode{Game.Event.trigger}) в методе перемещения камеры, где в качестве аргумента передадим саму камеру. Будем подразумевать, что при вызове события, вызовется его обработчик с аргументом, переданным в функции вызова:

	\begin{figure}[H]
\begin{lstlisting}[caption=Пример кода реализации описанного алгоритма событий]
in Game.init:
	sphere = Game.HyperSphere(Point(0, 0, 0), Vector(1, 0, 0))

	def handler(camera: Game.Entity):
		sphere.set_position(camera.get_position())

	Game.Event.add("OnMainCameraMove")
	Game.Event.handle("OnMainCameraMove", handler)

in Game.camera.movement:
	Game.Event.trigger("OnMainCameraMove", camera)
\end{lstlisting}
	\end{figure}


\subsection{Игровые параметры}
	
	Рассмотрим концепцию параметров игры. Введем в игру конфигурационный файл, содержащий в себе базовую информацию для настроек игры. К настройкам игры можно отнести управление, дальность прорисовки, размеры экрана, и другие значения. 

	Конфигурационный файл можно хранить в любом удобном формате. Все константные значения можно на данном этапе перенести в этот файл, поместив его в папку \inlinecoden{config/}, с названием, к примеру, \inlinecode{default.*} или \inlinecode{autoexec.*} (файл с расширением <<удобного>> формата).

	В концепцию параметров игры включим возможность подключать несколько конфигурационных файлов, которые будут вызываться последовательно. Для них разрешается перезаписывать данные, которые уже были записаны в других файлах.
	

\subsection{Класс EventSystem}
	
	\noindent Класс, отвечающий за реализацию и обработку события. Является независимым классом.

	\noindent Инициализация:
	\begin{enumerate}
		\item \inlinecode{EventSystem()} -- класс системы событий. Содержит в себе информацию о существующих событиях в игре.
	\end{enumerate}

	\noindent Реализуемые поля:
	\begin{enumerate}
		\item \inlinecode{events(dict[str: list[Callable]])} -- таблица информации о событиях. Ключем является имя события, значением -- массив вызываемых функций, каждая из которых принимает одно и то же число аргументов;
	\end{enumerate}

	\noindent Реализуемые методы:
	\begin{enumerate}
		\item \inlinecode{add(name: str)} -- метод, добавляющий в систему событий новое имя для события;
		\item \inlinecode{remove(name: str)} -- метод, удаляющий из системы событий заданное имя;
		\item \inlinecode{handle(name: str, function: Callable)} -- метод, добавляющий функцию-обработчик в массив функций для заданного имени события;
		\item \inlinecode{remove_handled(name: str, function: Callable)} -- метод, удаляющий функцию-обработчик из массива функций для заданного имени события;
		\item \inlinecode{trigger(name: str, *args)} -- метод, вызывающий в пространстве событий все функции-обработчики для заданного имени события с переданными аргументами;
		\item \inlinecode{get_handled(name: str)} -- метод, возвращающий все функции-обработчики для заданного события.
	\end{enumerate}

	\noindent Перегружаемые операторы:
	\begin{enumerate}
		\item \inlinecode{EventSystem[name: str]} (операция, возвращающая все функции-обработчики для заданного события, не имеет возможности установки значений).
	\end{enumerate}


\subsection{Класс Game.Configuration}

	Класс, содержащий информацию о значениях параметров, подключенных из отдельного файла. Параметры можно редактировать и подключать во время исполнения.

	\noindent Инициализация:
	\begin{enumerate}
		\item \inlinecode{Game.Configuration()} -- общий класс конфигурации игры;
		\item \inlinecode{Game.Configuration(filepath: str)} -- класс конфигурации игры, содержащий в себе информацию о константах или переменных из файла.
	\end{enumerate}

	\noindent Реализуемые поля:
	\begin{enumerate}
		\item \inlinecode{filepath(str)} -- строка пути к конфигурационному файлу. Может быть пустой строкой;
		\item \inlinecode{configuration(dict[str: any])} -- таблица подгруженных значений переменных из конфигурационного файла. В случае отсутствия файла, подгружает значение по умолчанию.
	\end{enumerate}

	\noindent Реализуемые методы:
	\begin{enumerate}
		\item \inlinecode{set_variable(var: str, value: any | None) -> None} -- метод, устанавливающий (или удаляющий) значение переменной в текущей конфигурации;
		\item \inlinecode{get_variable(var: str) -> any} -- метод, возвращающий значение переменной в текущей конфигурации;
		\item \inlinecode{execute_file(filepath: str)} -- метод, обновляющий значения переменных, подгруженных из файла;
		\item \inlinecode{save(filepath: str)} -- метод, сохраняющий конфигурацию в файл. В случае, если файл не указан, сохраняет в файл с именем файла из поля \inlinecode{filepath} текущей конфигурации. В случае, если файлы не указаны как в аргументе, так и в поле, вызывать ошибку невозможности сохранения файла.
	\end{enumerate}

	\noindent Перегружаемые операторы:
	\begin{enumerate}
		\item \inlinecode{Game.Configuration[var: str]} (оператор обращения (и присваивания) параметру значений).
	\end{enumerate}


\subsection{Класс Game}
	Для класса \inlinecode{Game} необходимо добавить в инициализацию систему событий, которая в дальнейшем будет использоваться для взаимодействия между сущностями.

	\noindent Инициализация:
	\begin{enumerate}
		\item \inlinecode{Game(cs: CoordinateSystem, es: EventSystem, entities: EntitiesList)} -- класс игры, содержащий в себе базовую информацию в виде системы координат, подключенной системы событий и списка сущностей.
	\end{enumerate}

	\noindent Реализуемые поля:
	\begin{enumerate}
		\item \inlinecode{es(EventSystem)} -- реализуемая система событий в данном экземпляре игры;
	\end{enumerate}

	\noindent Реализуемые методы:
	\begin{enumerate}
		\item \inlinecode{get_event_system() -> EventSystem} -- метод, возвращающий объект класса системы событий для данного экзмепляра игры;
		\item \inlinecode{apply_configuration(configuration: Game.Configuration)} -- метод, применяющий параметры конфигурационного файла в игре (изменяет переменные).
	\end{enumerate}


\subsection{Класс Game.Console[\textit{Game.Canvas}]}
	
	Введем класс для отрисовки изображения в консоли. Реализацию можно проводить на любом доступном движке консольной отрисовке, например, на движке \inlinecode{ncurses}.

	\noindent Реализуемые поля:
	\begin{enumerate}
		\item \inlinecode{Game.Console.charmap(list[str]))} -- список символов, расположенных в порядке увеличения/уменьшения заполняемого символом объема.
		Классический пример:
\begin{lstlisting}
charmap = ".:;><+r*zsvfwqkP694VOGbUAKXH8RD#$B0MNWQ%&@"
\end{lstlisting}
		Также, эти символы можно перенести в конфигурационный файл, опционально.
	\end{enumerate}

	\noindent Реализуемые методы:
	\begin{enumerate}
		\item \inlinecode{draw() -> None} -- метод, производящий отрисовку в классе \inlinecode{Game} известной матрицы \inlinecode{distances} в консоль с использованием библиотеки отрисовки и символов, заданных в данном классе.
	\end{enumerate}


\subsection{Дополнительно}

	В данной работе требуется реализовать также управление камерой в пространстве. Рассматривать будем трехмерное пространство, движение возможно по шести направлениям -- вперед, назад, влево, вправо (без вращения камеры), вверх и вниз (клавишами \inlinecoden{space}, \inlinecoden{ctrl}). Поворот камеры осуществляется в двумерном пространстве -- вращение в горизонтальной плоскости \( XY \), и вертикальной, пересекающей камеру по центру (в плоскости направляющего вектора и вектора, отвечающего за ось $z$). Для поворота можно обозначить клавиши <<стрелочки>> -- повороты влево, вправо, вверх и вниз.


\subsection{Этапы реализации}

	Зафиксируем представленные классы в поэтапной реализации движка:
	\begin{enumerate}
		\item Реализуем класс системы событий, способный обрабатывать события для игры, и пропишем его при инициализации игры;
		\item Добавим в игру понятие параметров, в которые запишем все возможные константы или переменные (параметры) игры;
		\item Добавим класс отрисовки в консоль на основе класса <<полотна>> игры;
		\item Добавим возможность управления камерой с помощью клавиатуры. Движения должны обрабатываться системой событий, когда игра отлавливает нажатия на клавиши, передает событие перемещения, а камера обрабатывает событие, совершая действия.
	\end{enumerate}